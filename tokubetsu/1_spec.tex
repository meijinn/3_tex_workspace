\mysection{仕様と実現方法}
\begin{enumerate}
  \item たまにある単純作業の繰り返しの作業は何か \\
  光学式測距センサ(LiDAR)で得た距離データ(センサデータ)のばらつきを統計的に処理したい。
  ロボットの自律移動を行うために、ロボットの自己位置・速度の推定を行う必要がある。
  一例として、LiDARを用いてランドマークとの距離を計測し、自己位置を得る手法が提案されている。
  ロボットに搭載するセンサデータは、センサ内部の状態や周囲の環境等、さまざまな影響を受けて出力される。
  \item どのようなデータを可視化したいか\\
  センサデータに含まれる誤差(ノイズ)の傾向を可視化したい。

  \item どのような機能があればそれが実装できそうか\\
  統計処理で扱う関数やモジュールと、それを可視化するためのツールがあれば実装できる。
  センサデータを解析するために、pythonにおいて、以下に示したツールを用いることが想定される。
    \begin{itemize}
      \setlength{\parskip}{0cm} % 段落間
      \setlength{\itemsep}{0cm} % 項目間
      \item 距離データが格納されたファイルの読み込み
      \item 配列等への格納
      \item 確率分布、平均、分散、標準偏差、正規分布を扱うためのモジュール
      \item 数値の可視化 (グラフ化ツール)
    \end{itemize}
  順にこれらを実装することで、測距データに含まれるノイズの傾向が明らかになる。
\end{enumerate}