\documentclass[a4paper,12pt]{jreport}
\usepackage{bm}
\usepackage[dvipdfmx]{graphicx}
\usepackage{ascmac}

\usepackage{bm}
\usepackage[dvipdfmx]{graphicx}
\usepackage{amssymb,amsmath}
\usepackage{ascmac}
\usepackage{float}
\usepackage{setspace}
\usepackage[dvips,usenames]{color}
\usepackage{colortbl}
\usepackage{algorithm}
\usepackage{algorithmic}
\usepackage{setspace}

\definecolor{bl}{rgb}{0.94,0.97,1}
\definecolor{gr}{rgb}{0.5,0.5,0.5}
\makeatletter
\def\section{\newpage\@startsection {section}{1}{\z@}{2.3ex plus -1ex minus -.2ex}{2.3 ex plus .2ex}{\Large\bf}}
\makeatother

\setlength{\textwidth}{160truemm}
\setlength{\textheight}{240truemm}
\setlength{\topmargin}{-14.5truemm}
\setlength{\oddsidemargin}{-0.5truemm}
\setlength{\headheight}{0truemm}
\setlength{\parindent}{1zw}

\setstretch{1.0}
% ヘッダーとフッターの設定
\usepackage{fancyhdr}
\rhead{\leftmark}
\chead{}
\lhead{\rightmark}
\cfoot{\thepage}

\rfoot{}


\title{卒論・修論を\LaTeX で書く}
\author{弘前大学理工学部地球環境防災学科\\
学籍番号 名前}
\date{2020年吉日}

\begin{document}
\maketitle
\tableofcontents

\pagenumbering{roman}
\newpage

\clearpage
\pagenumbering{arabic}

\pagestyle{fancy}
\setlength{\headheight}{5truemm}
\chapter{はじめに}

最初はイントロ的なことを書く。
\section{現状と問題点}

最近の現状と問題点とか。

\section{解決策の提案}

こうしたらいい,とか。

\section{数式の書き方}

アインシュタイン方程式は以下の通りである。
\begin{equation}
R_{\mu\nu} - \frac{1}{2} g_{\mu\nu} R =
\frac{8\pi G}{c^2} T_{\mu\nu}
\end{equation}

\chapter{つぎに}

この辺から本番。

\section{文献の引用の仕方}

データは参考文献\cite{rika} にあったものを使った.
この文献\cite{ten}も参考にした。

\section{図の挿入の仕方}


\chapter{最後に}

結論とか,まとめとか。
最後にいうのもなんだが,ベクトルの書き方。
\begin{itemize}
\item 普通の$\alpha$は\verb|\alpha|で書く。
\item \verb|$\vec{\alpha}$| で $\vec{\alpha}$
\item \verb|\usepackage{bm}| している場合は
\verb|$\bm{\alpha}$| で $\bm{\alpha}$
\item 並べると,$\alpha$, $\vec{\alpha}$, $\bm{\alpha}$
\end{itemize}

\chapter*{謝辞}

謝辞には第何章とかの番号をつけなくてもよいので,そんなときは,
\verb|\chapter*{ }| という具合に書きます。

みなさん,ありがとう.(普通の人が見るのは,イントロと謝辞だけ...
という説もあるから,忘れないで書く.)

\appendix
\chapter{付録があるときは}
プログラム文とかを書いてページ数を稼ぎたいときは,
以下のようにしてみます。

\begin{verbatim}
#include <iostream>
using namespace std;
int main() {
for(int i = 1; i <= 5; i++) {
cout << "こんにちは, C++ の世界! " << i << endl;
}
return 0;
}
\end{verbatim}
\verb|\usepackage{ascmac}|して\verb|screen| 環境を使うと,枠がつきます。
\begin{screen}
\begin{verbatim}
#include <iostream>
using namespace std;
int main() {
for(int i = 1; i <= 5; i++) {
cout << "こんにちは, C++ の世界! " << i << endl;
}
return 0;
}
\end{verbatim}
\end{screen}

\begin{thebibliography}{99}
  \bibitem{rika} 国立天文台編,理科年表 (丸善)
  \bibitem{ten} 天文年鑑,誠文堂新光社。
  \end{thebibliography}
  
  \end{document}