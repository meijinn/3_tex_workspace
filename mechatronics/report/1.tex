\mysection{演習1}
\subsection{実行プログラム}
実行プログラムをソースコード\ref{s1}に示す。
\begin{lstlisting}[caption=演習1のプログラム,label=s1]
#include<stdio.h>
int main()
{
  printf("Hello/n");
  return 0;
}
\end{lstlisting}

\mysubsection{実行結果}
JTWのコンソール上にHelloと表示される。
\mysection{演習2}
\subsection{実行プログラム}
実行プログラムをソースコード\ref{s2}に示す。
\begin{lstlisting}[caption=演習2のプログラム,label=s2]
#include <stdio.h>
#include <process.h>

int main()
{
	int i;
	FILE *fp;

	if((fp = fopen("test.dat", "wt")) == NULL){
		printf("Cannot Open test.dat!!\n");
		exit(1);
	}
	
	for(i = 0; i < 100; i++){
		fprintf(fp, "%3d\n", i);
	}
	fclose(fp);
	return 0;
}  
\end{lstlisting}

\mysubsection{実行結果}
ロボット上で得られたデータファイルをソースコード\ref{s3}に示す。0\verb|~|100までの数値が出力される。
\begin{lstlisting}[caption=ロボット上で得られたデータファイル(TEST.DAT),label=s3]
  0
  1
  2
  3
  4
  5
  6
  7
  8
  9
 10
~~~~~
 90
 91
 92
 93
 94
 95
 96
 97
 98
 99
..................
\end{lstlisting}