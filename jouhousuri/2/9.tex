\mysection{9ページ目翻訳}

\begin{enumerate}
  \setcounter{enumi}{5}
  \item $m$と$b$をともに0でない非負の整数とし、
  $m$個の偶数頂点と$n$個の奇数頂点を持つグラフが常に存在するとは限らないことを示せ。
  ただし、$n$が偶数であることが必要な場合は、そのようなグラフが存在することを示せ。
  \item \begin{enumerate}
    \item 各$r, 0 \leq r < 8$について、位数8の$r$-regularグラフを構築せよ。
    \item (a)で構築した各グラフの補数を調べよ。
    \item (c) $G$が正則グラフであれば、$G$は正則であることを示せ。
  \end{enumerate}
  \item トムとその妻が、3組の注文のある夫婦と一緒にパーティーに参加したとする。
  何度か握手が行われた。同じ人と2回以上握手する人はいませんでした。
   すべての握手が終わった後、トムは妻を含む各人に何回握手したかを尋ねた。各自が違う答えをした。
  \begin{enumerate}
    \item トムは何人と握手したのでしょうか?
    \item 奥さんは何人と握手しましたか?
  \end{enumerate}
  \item 次数14、サイズ25のグラフ$G$のすべての頂点は次数3または5である。$G$は次数3の頂点をいくつ持っているか?
  \item 次数7、サイズ10のグラフ$G$は、次数$a$の頂点が6個、次数$b$の頂点が1個ある。$b$は何個か?
  \item 次数4のグラフは、次数3の頂点が3つ、次数1の頂点が1つあることがあるか?
\end{enumerate}

\begin{description}
  \item[1.3 同型のグラフ] 
\end{description}

同じグラフを表す2つの図が、全く違うものに見えることがあります。
このことは、図1-3や図1-4ですでに確認済みです。しばしば、2つのグラフ$G_1$と$G_2$が実際に同じグラフであるかどうかを知ることが重要である。
直感的には、それぞれを描画(または再描画)してもう一方のグラフを得ることができれば、
それらは同じであると言える。
この考えを定式化するために、同型という概念を導入する。
2つのグラフ$G_1$、$G_2$は、$V(G_1)$から$V(G_2)$への一対一の関数$\phi$が存在し、$\phi(u) \phi(v) \in E(G_1)$の場合に限り、
$uv$が$E(G_1)$に存在する場合、同型であるとする。(表記を簡略化するため、$u$の像を$\phi(u)$ではなく、$\phi(u)$と書く。)


この関数$\phi$は同型と呼ぶ。
$G_1$と$G_2$が同型である場合、$G_1 \cong G_2$と書く。

図1-9のグラフG1とG2は、関数$\phi:V(G_1) \rightarrow V(G_2)$ のため、同型である。

$\phi v_i = u_i (i = 1,2,\ldots,5)$で定義される$V(G_1)$は同型である。

したがって、グラフ$G_2$は、$G_1$が得られるように描画することができ、
ここで、$u_1$は、すべてのについて$v_1$に置き換えられる。$i (1 \leq i \leq 5)$