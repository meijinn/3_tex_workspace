\mysection{60ページ目翻訳}

具体的には、$\deg v_i = d > 0$ とします。 $T(i,2) = j$ の場合、$j > 0$ となり、
$v_i$ に隣接する最初の頂点は $T(j,1)$ になります。 $v_i$ に隣接する $d$ 頂点は、実際には $T(n, 1)$ ($n = j, j+1, \dots, j+d-1$) です。 $T(j+d-1, 2) = 0$ のとき、$v_i$ の隣接リストが完成します。

たとえば、図 2-8 のグラフ $G$ の頂点 $v_2$ について考えてみましょう。 エントリ $T(2, 1)$ は空白ですが、$T(2,2)$ はポインタ 9 です。 そこで $T(9,1)$ を見てみましょう。 $T(9,1)$は1なので、頂点$v_2$は$v_1$に隣接します。 ここで $T(9,2)$ はポインタ 10 です。エントリ $T(10, 1)$ は 3 です。 したがって、$v_2$ も $v_3$ に隣接しています。 エントリ $T(10, 2)$ はヌル ポインタ 0 です。

隣接リストを使用して $(p, q)$ グラフ $G$ を表現するには、$2p+4q$ の場所が必要です。 $G$ に含まれるエッジが比較的少ない場合、たとえば $q$ が $p$ 内で線形である場合、コンピューターでの $G$ の表現には $O(p)$ の位置が必要です。 これにより、$O(p^2)$ の位置を使用する隣接行列表現が改善されます。

これらの表現はダイグラフにも適用されます。 隣接行列
$V(D) = \{v_1, v_2, \dots, v_p \}$ を持つ $(p,q)$ 有向グラフ $D$ の $A = [a_{i,j}]$ は $p \ によって定義される p$ 行列の倍

もし...

有向グラフの隣接リストは、$i$ 番目の隣接リストで、$v_i$ が隣接する頂点が (番号順に) リストされることを除いて、グラフの隣接リストと同じです。 繰り返しますが、各リストは null ポインターで終わります。 $(p, q)$ ダイグラフの隣接リスト テーブルには $p+q$ 行があります。