\mysection{23ページ目翻訳}

証明: $e=uv$とし、逆に$e$がサイクル$C:u, v, w, ..., x, u$上にあるとする(つまり、$C$上で$w$が$v$に続き、$x$が$u$に先行する)。
グラフ$G-e$は$u-v$経路、すなわち$u, x, \cdots, w, v$を含むので、$u$は$v$に接続されていることになる。
$u_i$と$v_i$を$G-e$の任意の個の頂点とすると、$G-e$は$u-v$パスを含むことを示す。もし辺$e$が$P$上になければ、
$P$は$G-e$のパスであり、$u_j$は$G-e$で$v_i$に接続されている。ここで、辺$e$が$P$上にあるとすると、
経路$P$は$u_1, ., W, V, ..., V_i$または$W1, ..., V, u, ..., V1$で表すことができる。
しかし、そのとき、$w_1, u, x, w, v, v_i$または$u1, v, w, x, u, v_1$はそれぞれ、$G-e$における$u_j-v_i$歩道である。これは定理1.3により、$G-e$が$u_j-V_i$道を含むことを意味する。
したがって、$e$が$a$-cycleに属するなら、$G-e$は接続されており、$e$はブリッジではない。これにより、所望の矛盾が生じる。
逆に、$e = uv$が$G$のどのサイクルにも存在しない辺だとすると、$G-e$には$u-v$の経路が含まれない。
そうでない場合、$P$が$G-e$の$u-v$パスであれば、
$P$と辺$uv$を合わせて$e$を含むサイクルを生成し、さっきの仮定と矛盾する。

橋を含まないグラフは、まさにすべての辺がサイクル上に存在するグラフである。
次に、切断頂点を含まないグラフを考える。カット頂点を含まない非自明な連結グラフを非可分グラフと呼ぶ。
$G$を非自明な連結グラフとする。$G$のブロック$B$は、それ自身が非可分グラフであり、
この性質に関して最大であるGの部分グラフである。
ブロックは必然的に誘導部分グラフであり、グラフのブロックはグラフの辺集合の分割を作り出す。
すべての2つのブロックは最大で1つの頂点を共有する。さらに、2つのブロックが頂点を共有する場合、
これはカット頂点である。非可分なグラフは、必然的に1つのブロック、すなわちそれ自身しか持たない。
図1-19のグラフ$G$は3つのブロック、すなわち$({V、V2、V3})$を持つ、
$({v_3,v_4,v_5)、(v_5,v_6})$である。グラフ$G$とその3つのブロックは、図1-20に示すとおりである。