\mysection{7ページ目翻訳}

\begin{description}
  \item[証明] $G$の頂点の次数を合計するとき、
  各辺は、その2つの接続頂点それぞれについて1回ずつ数えられる。この定理には有用な帰結がある。

  \item[補題1.1] すべてのグラフは偶数個の奇数頂点を含む。 

  \item[証明] 偶数頂点の集合を$V_e$、
  サイズ$q$のグラフ$G$の奇数頂点の集合を$V_o$とすると、
  $$\sum_{v\in V(G)} \deg v = \sum_{v\in V_e} \deg v + \sum_{v\in V_o} \deg v = 2q,$$
  なので、この式の右辺の式はすべて偶数なので、左辺の和は偶数の項を含まなければならず、すなわち$|V_o|$のサイズは偶数である。
\end{description}

グラフ$G$は、$G$のすべての頂点が位数$r$を持つとき、\textbf{$r$-regular($r$-正則)}、または次数$r$の正則であるという。
図1-7には、$r = 0,1, \ldots 5$で次数6の$r$-regularグラフ$H$も示されている。
Gが次数$p$の$r$-regularグラフであれば、当然ながら$0 \leq r \leq p-1$となる。
しかし、$0 \leq r \leq p-1$であれば、次数$p$の$r$-正則グラフは存在しないことになる。