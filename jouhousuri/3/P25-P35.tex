\documentclass[10pt,a4j,dvipdfmx]{jarticle}
%---------------------------------------------------
\usepackage{hyperref}
\usepackage{pxjahyper}
\usepackage{bm}
\usepackage{graphicx}
\usepackage{amssymb,amsmath,mathtools}
\usepackage{ascmac}
\usepackage{float}
\usepackage{setspace}
\usepackage[dvips,usenames]{color}
\usepackage{colortbl}
\usepackage{algorithm}
\usepackage{algorithmic}
\usepackage{setspace}
\usepackage{subfigure}
\usepackage[deluxe,bold]{otf}
\usepackage[haranoaji]{pxchfon}
\usepackage{redeffont}
\usepackage{listings,jvlisting} %日本語のコメントアウトをする場合jvlisting(もしくはjlisting)が必要
\usepackage{booktabs}
%---------------------------------------------------
\definecolor{bl}{rgb}{0.94,0.97,1}
\definecolor{gr}{rgb}{0.5,0.5,0.5}
% \makeatletter
% \def\section{\newpage\@startsection {section}{1}{\z@}{2.3ex plus -1ex minus -.2ex}{2.3 ex plus .2ex}{\Large\bf}}
% \makeatother
%---------------------------------------------------
\setlength{\textwidth}{160truemm}
\setlength{\textheight}{240truemm}
\setlength{\topmargin}{-14.5truemm}
\setlength{\oddsidemargin}{-0.5truemm}
\setlength{\headheight}{0truemm}
\setlength{\parindent}{1zw}
\setlength{\abovedisplayskip}{-2pt} % 数式上部のマージン
\setlength{\belowdisplayskip}{0pt} % 数式下部のマージン
%---------------------------------------------------
\setstretch{1.2}
%---------------------------------------------------
\renewcommand{\subfigtopskip}{5pt}	% 図の上の隙間。上図の副題と下図の間。
\renewcommand{\subfigbottomskip}{0pt} % 図の下の隙間。副題と本題の間。
\renewcommand{\subfigcapskip}{-6pt}	% 図と副題の間
\renewcommand{\subcapsize}{\scriptsize} % 副題の文字の大きさ
\newcommand{\mysection}[1]{\newpage\vspace{-20pt}\section{#1}}
\newcommand{\mysubsection}[1]{\vspace{-20pt}\subsection{#1}}
\newcommand{\mysubsubsection}[1]{\vspace{-10pt}\subsubsection{#1}}
%---------------------------------------------------
% ヘッダーとフッターの設定
\usepackage{fancyhdr}
\rhead{\leftmark}
\chead{}
\lhead{\rightmark}
\cfoot{\thepage}

\rfoot{}
\begin{document}
%---------------------------------------------------
\setlength{\abovedisplayskip}{1.5pt} 
\setlength{\belowdisplayskip}{0pt}
%---------------------------------------------------
%ここからソースコードの表示に関する設定
\lstset{
  basicstyle={\ttfamily},
  identifierstyle={\small},
  commentstyle={\smallitshape},
  keywordstyle={\small\bfseries},
  ndkeywordstyle={\small},
  stringstyle={\small\ttfamily},
  frame={tb},
  breaklines=true,
  columns=[l]{fullflexible},
  numbers=left,
  xrightmargin=0zw,
  xleftmargin=3zw,
  numberstyle={\scriptsize},
  stepnumber=1,
  numbersep=1zw,
  lineskip=-0.5ex
}
%ここまでソースコードの表示に関する設定
%---------------------------------------------------
\pagenumbering{arabic}
\pagestyle{fancy}
\setlength{\headheight}{5truemm}

% \mysection{20ページ目翻訳}

1.6 連結グラフ

本節では、グラフの構造についての学習を始めます。
グラフ$G$における歩道とは、交互に並ぶ

$$W: v_0, e_1, v_1, e_2, v_2,\dots, v_{n-1},e_n, v_n (n\geq 0)$$

頂点と辺のうち、頂点で始まり頂点で終わるもので、
$i = 1, 2,\cdots, n$ に対して $e_i = v_{i-1}v_i$ となるものです。

$W$は$v_0$から始まり$v_n$で終わるので、
$W$を$v_o-v_n$道と呼ぶ。また、$W$は$n$個の(必ずしも異なる)辺に
出会うので、$W$は長さ$n$を持つとも言われます。
長さ0の歩道を自明な歩道と呼ばれます。
図1-17のグラフ$G$において

$$u, e_1, v, e_3, w, e_6, x, e_4, u, e_1, v, e5, y$$

は長さ6の歩道であり、歩道に現れる頂点が歩道の辺を決めるので、
辺は省略することができます。したがって、辺(1.2)はより簡単に
$u, v, w, x, u, v, y$と表現することができます。

グラフ理論でよく登場する歩道には、いくつかの特殊な種類があります。
小道は辺が繰り返されない歩道であり、
道は頂点が繰り返されない歩道です。したがって、
道は小道ですが、すべての小道が道であるわけではありません。
図1-17のグラフ$G$において、$x, w, v, u, w, y$という歩道は
道ではない小道であり、$u, x, w, v, y$は道です。

\begin{description}
  \item[定理1.3] グラフ内のすべての$u-v$歩道は、$u-v$道を含む。\\
  証明: グラフ$G$の$u-v$歩道を$W$とする。$u=v$とすると、
  自明な$u-u$道が結果を与えるので、$u \neq v$と仮定すればよい。
  $Wを : u = uo, u1, ... u_n=v$(一つの頂点が複数のラベルを受け取ることもありうる)と仮定する。
  $W$に$G$の頂点が2回以上登場しない場合、$W$自体が$u-v$道となる。
  そうでない場合は、$W$に2回以上現れる$G$の頂点が存在する。
  $u_j=u_j$となるような$i$と$j$を異なる正整数、$i<j$とする。
  $W$から、$u_i, u_{i+j}\cdots u_{j-1}$の項を削除すると、
  長さが$W$の長さより小さい$u-v$歩道$W_1$が得られる。
  $W_1$に$G$の頂点が2回以上出現しない場合、$W_1$は$u-v$の道である。
  それ以外の場合は、最終的に$u-v$経路に到達するまで、上記の手順を適用する。
\end{description}
% \mysection{21ページ目翻訳}

閉路とは、$n>3$, $v_o=v_n$, $n$個の頂点$v_1, v_2,\dots, v_n$
が別々である歩行$v_o, v_1, \cdots, v_n$のことである。
長さ$n$の閉路は$n$サイクルと呼ばれる。$u-v$の歩道は、
$u=v$なら閉歩道、$u\neq v$なら開歩道と言う。 
従って、閉路は閉歩道になり、自明でない道は開歩道になる。

自明でない閉歩道のことを回路という。
したがって,すべての閉路は回路であるが,
回路は閉路である必要はない。
(回路と閉路に関する定理1.3の類推は問題9を参照。)
図1-17のグラフに戻り、$u, w, y, v, u$は閉路(4サイクル)であり、
$u, x, w, v, y, w, u$は閉路でない回路であることに注意せよ。

グラフ$G$の頂点を$u$と$v$とし、
$G$に$u-v$の経路が含まれる場合、$u$は$v$に接続されているという。
グラフ$G$の頂点のすべての組$u,v$に対して$u$が$v$に接続されている場合、
グラフ$G$自体は接続されていることになる。
したがって、グラフ$G$は、$u-v$のパスが存在しない2つの頂点$u$と$y$が
存在する場合に切断される。このとき、自明グラフは連結されている。
グラフ$G$の部分グラフ$H$は、$H$が$G$の最大連結部分グラフであるとき、
$G$の構成要素である。

~に接続されている という関係は、グラフ$G$の頂点の集合上の同値関係である。
(問題2参照)とであり、したがって、$V(G)$の分割$V_1、V_2,\dots, V_k$を生じさせる。
グラフ$G$の構成要素の数を$k(G)$で表すと、
$G$の$1\leq i\leq k$の部分グラフ$\left\langle V_i \right\rangle _G$が$G$の構成要素である。
したがって、グラフ$G$は$k(G)=1$である場合にのみ連結される。図1-18のグラフ$G_1$は連結しているが、
$G_2$は$k(G_2)=4$で切断されている。
% \mysection{22ページ目翻訳}

1.7 切断頂点と橋
連結グラフの頂点、辺、部分グラフには、
特筆に値する次のような種類があります。
これらについては、本節で説明する。

グラフ$G$の頂点$v$は、$k(G-v) > k(G)$のとき、
切断頂点と呼ばれる。
つまり、$G-v$が切断されている場合、
$v$は連結グラフ$G$の切断点となる。
図1-19のグラフGの頂点$v_3$と$v_s$は、
このグラフの唯一の切断点である。
ここで、切断頂点の類似物である辺について考えてみる。
グラフ$G$の辺$e$は、$k(G-e) > k(G)$のとき、
橋と呼ばれます。
したがって、連結グラフ$G$の辺$e$は、$G-e$が切断されていれば
ブリッジである。図1-19のグラフ$G$の辺$e=v_5v_6$は、
このグラフの唯一のブリッジである。
$v$が連結グラフ$G$の切断頂点である場合、
$G-v$には2つ、あるいはより多くの構成要素を含む。
しかし、$e$が$G$のブリッジである場合、
$G-e$はちょうど2つの構成要素を含む。
次の定理は、グラフの辺のうち、橋であるものを特徴づけるものである。
\begin{description}
  \item[定理1.4.] 連結グラフ$G$の辺$e$は、$e$が$G$のサイクル上にない場合に限り、$G$のブリッジとなる。
\end{description}
% \mysection{23ページ目翻訳}

証明: $e=uv$とし、逆に$e$がサイクル$C:u, v, w, ..., x, u$上にあるとする(つまり、$C$上で$w$が$v$に続き、$x$が$u$に先行する)。
グラフ$G-e$は$u-v$経路、すなわち$u, x, \cdots, w, v$を含むので、$u$は$v$に接続されていることになる。
$u_i$と$v_i$を$G-e$の任意の個の頂点とすると、$G-e$は$u-v$パスを含むことを示す。もし辺$e$が$P$上になければ、
$P$は$G-e$のパスであり、$u_j$は$G-e$で$v_i$に接続されている。ここで、辺$e$が$P$上にあるとすると、
経路$P$は$u_1, ., W, V, ..., V_i$または$W1, ..., V, u, ..., V1$で表すことができる。
しかし、そのとき、$w_1, u, x, w, v, v_i$または$u1, v, w, x, u, v_1$はそれぞれ、$G-e$における$u_j-v_i$歩道である。これは定理1.3により、$G-e$が$u_j-V_i$道を含むことを意味する。
したがって、$e$が$a$-cycleに属するなら、$G-e$は接続されており、$e$はブリッジではない。これにより、所望の矛盾が生じる。
逆に、$e = uv$が$G$のどのサイクルにも存在しない辺だとすると、$G-e$には$u-v$の経路が含まれない。
そうでない場合、$P$が$G-e$の$u-v$パスであれば、
$P$と辺$uv$を合わせて$e$を含むサイクルを生成し、さっきの仮定と矛盾する。

橋を含まないグラフは、まさにすべての辺がサイクル上に存在するグラフである。
次に、切断頂点を含まないグラフを考える。カット頂点を含まない非自明な連結グラフを非可分グラフと呼ぶ。
$G$を非自明な連結グラフとする。$G$のブロック$B$は、それ自身が非可分グラフであり、
この性質に関して最大であるGの部分グラフである。
ブロックは必然的に誘導部分グラフであり、グラフのブロックはグラフの辺集合の分割を作り出す。
すべての2つのブロックは最大で1つの頂点を共有する。さらに、2つのブロックが頂点を共有する場合、
これはカット頂点である。非可分なグラフは、必然的に1つのブロック、すなわちそれ自身しか持たない。
図1-19のグラフ$G$は3つのブロック、すなわち$({V、V2、V3})$を持つ、
$({v_3,v_4,v_5)、(v_5,v_6})$である。グラフ$G$とその3つのブロックは、図1-20に示すとおりである。
% \mysection{24ページ目翻訳}

グラフ$G$のブロックのうち、$G$のカット頂点をちょうど1つ含むものを、$G$のエンドブロックと呼ぶ。
グラフのエンドブロックに関する次の定理は、後で役に立つので、証明なしで述べる。
\begin{description}
  \item[定理1.5.] $G$を少なくとも1つのカット頂点を持つ連結グラフとする。すると、$G$は少なくとも2つのエンドブロックを持つ。 
\end{description}
% \mysection{25ページ目翻訳}

1.8 特殊グラフ

グラフのあるクラスは頻繁に参照されるため、
特別な名前がつけられている。
まず、すべての可能な辺を持つグラフを考えることから始める。
すべての異なる2つの頂点が隣接する次数$p$のグラフは完全グラフであり、
Kpと表記される。
したがって、次数pの完全グラフは$(p - 1)$正則であり、
サイズ$q = (p/2) = p(p - 1)/2 $を持つ。
グラフ$K_p(1\leq p \leq 6)$を図1-21に示す。

1.6節で、パスとサイクルの概念を紹介しました。
順序$n \geq 1$のグラフでそれ自体がパスであるものを
順序$n$のパスと呼び、$P_n$と表記し、順序$n \geq 3$のグラフでそれ自体が
サイクルであるものを$C_n$と表記し、$n$サイクルと呼ぶ。
図1-22は3つのパスと3つのサイクルを示しています。
パスはその順序が偶数か奇数かによって長さが奇数か偶数になり、
サイクルはその長さが奇数か偶数かによって長さが偶数になる。
グラフ$G$は、$V(G)$が2つの(空でない)グラフに分割できるとき、二部構成である。
の頂点と$V$の頂点を結ぶ$G$のすべての辺が、$V$と$V$の部分集合になるようにする。
\mysection{26ページ目翻訳}
図 1-23a のグラフは、$V(G)$ が部分集合 $V_1 = \{ v_1, v_6\} $ と $V_2 = \{ v_2, v_3, v_4, v_5\} $に分割され、
$G$の各辺がこれらの集合の頂点を結ぶため、2部グラフになります。このグラフは図 1-23b に再描画されており、
$V_1$ が上部に、$V_2$ が下部にあり、$G$ が2部構成である理由をより明確に示しています。
二部グラフとサイクルの間には特別な関係があります。

証明

まず、$G$ が2部グラフであると仮定します。 次に、$V(G)$ は、$G$の辺が$V_1$の頂点と$V_2$の頂点に結合するように、
2つの空でない部分集合$V_1$と$V_2$に分割できます。$G$にその$n$サイクル $C: v_1, v_2, \dots, v_n, v_1$が含まれているとします。 $n$が偶数であることを示します。 
$v_1 \in V_1$ とします。 辺$v_1v_2$は$V_1$の頂点と $V_2$ の頂点を結合するため、必然的に $v_2 \in V_2$です。
同じ理由で、$v_3 \in V_1, v_4 \in V_2$ などです。 $v_nv_1$ は $G$ と $v_1 \in V_1$ の辺であるため、$v_n \in V_2$ に従い、$n$は偶数になります。
逆に、$G$が奇数サイクルのないグラフであるとします。 私たちの目標は、$G$が二部であることを示すことです。 最初に$G$が連結されていると仮定し、
$v_1$を$G$の頂点とします。$G$は連結されているため、$G$の各頂点$v$に対して$G$に$v_1-v$基本道が存在します。
\mysection{27ページ目翻訳}
もちろん、与えられた頂点$v$に対して、
$G$に複数の$v_1-v$基本道が存在する場合があります。
そのような場合、存在する必要がある最短の長さのものを選択します 
(最短の長さが複数ある場合は、いずれかを選択します。 )
$V(G)$の部分集合$V_1$を$v_1$と$G$ の各頂点$v$で構成されるように定義し、
$G$の最短の$v_1-v$基本道が偶数の長さと$V_2$の頂点を持つようにします。
これが起こらないと仮定します。 次に、$G$のいくつかの辺$e$が
$V_2$の2つの頂点を結合する必要があります。
$e$が$V_2$の2 つの頂点$v_j$と$v_k$を結ぶとします。
$v_j$と$v_k$は$V_2$に属しているため、
それぞれの最短$v_1-v_j$基本道と$v_1-V_2$の$v_k$です。
$v_j$と$v_k$は$V_2$に属しているため、
最短の$v_1-v_j$基本道と$v_1-v_k$基本道はそれぞれ奇数の長さになります。
$P$を最短$v_1-v_j$基本道、$Q$を最短$v_1-v_k$基本道とする。
次に、$P, Q$および辺$v_jv_k$によって、奇数の長さの閉じた道が生成されます。 
しかし、この結果は、問題セット 1.6 の問題 10 によって、$G$ が奇数サイクルを持ち、矛盾を与えることを意味します。
したがって、$V_2$の2つの頂点は隣接しません。 同様の引数は、$V_1$の2つの頂点が隣接していないことを示しています。
これで、$G$は2部であると結論付けることができます。

$G$が切断されている場合、$G$には2つ以上のコンポーネント
$(G_1, G_2, \dots, G_n など)$ があります。$G$の各サイクルは偶数であるため、
$G$のすべての要素の各サイクルは偶数です。 
要素は接続されているため、前の引数は各要素が2部構成であることを示しています。
次に、たとえば、$V(G_1)$を部分集合$U_1$と$W_1$に分割して、$G_1$の各辺が
$U_1$の頂点と$W_1$の頂点に結合するようにすることができます。
一般に、$G_1$の各エッジは、$U_1$の頂点と$W_1$の頂点を結合します。
一般に、各$i (1 \leq i \leq n)$ について、$V(G_i)$を部分集合
$U_i$と$W_i$に分割して、$G_i$の各辺が$U_i$の頂点と$U_i$の頂点と$W_i$の頂点を
結ぶようにすることができます。 したがって、$G$の各辺は、$\bigcup ^n_{i=1}$の頂点$U_i$と$\bigcup ^n_{i=1}$の頂点$W_i$を結び、
$G$が2部であることを意味します。
グラフ$G_i$が2部構成の場合、$V(G)$ を部分集合$V_1$と$V_2$に分割して、
$G$のすべての辺が$V_1$の頂点と$V_2$の頂点に結合できることがわかっています。
ただし、これは、$V_1$の各頂点が$V_2$のすべての頂点に隣接していることを意味するものではありません。
この場合、$G$は完全な2部グラフと呼ばれます。$|V_1| = m$、および $|V_2| = n$ とする。
このグラフを $K_{m,n}$ とする。 グラフ $K_{1,n}$ (または $K_{n,1}$) はスターと呼ばれます。 $K_{1,1} \cong K_{2}, K_{1, 2} \cong P_3$ および $K_{2,2} \cong C_4$ であることに注意してください。

いくつかの完全な 2 部グラフを図1-24に示します。
2部グラフは自然な方法で$n$部グラフに拡張できます。 
$n \geq 2$ の場合、$V(G)$ が$n$個の空でない部分集合
$V_1, V_2, \dots, V_n$ に分割され、$G$の辺が同じ集合内の頂点に結合しない場合、
グラフ$G$は$n$部分グラフです。 集合$V_1, V_2, \dots, V_n$ は、$G$ の辺が同じ集合内の頂点を結合しないようにします。
集合$V_1, V_2, \cdots, V_n$は、$G$の分割集合と呼ばれます。

$G$が部分セット$V_1, V_2, \dots, V_n$ を持つ $n$-部分グラフで、$V_i$ のすべての頂点が$V_j$のすべての頂点に結合されている場合 
$(1 \leq  i \leq  j \leq n)$, $G$は完全グラフと呼ばれます。$n$部分グラフ$|V_i| = p_i$の場合、$i = 1, 2, \dots, n$ の場合、$G$を$K_{p1,p2,\dots, p_n}$で表します。
これらのグラフは、完全多部グラフとも呼ばれます。図1-25にいくつかの例を示します。
他のグラフからグラフを作成する方法はいくつかあります。 この一例、つまり補数についてはすでに見てきました。
この場合、操作は1 つのグラフに対してのみ実行され、新しいグラフが生成されます。 
次に、2つ以上のグラフに対する操作を考えます。
\mysection{29ページ目翻訳}

$G_1$と$G_2$を頂点非接続グラフとする。
したがって、$G_1\cup G_2$ で表される$G1$と$G_2$の結合は、$V(G_1 \cup G_2)$ と $E(G_1 \cup G_2) = E(G_1) \cup E(G_2)$を持つグラフです。
$G_1 \cong G_2 \cong G$ の場合、$G_1 \cup G_2$を
$2G$と書きます。 一般に、$G_1, G_2, \dots ,G_n$ が $G$ に同形な一対一頂点非接続グラフの場合、$G_1 \cup G_2 \cup \dots \cup G_n$ を $nG$ と書きます。
図1-26 は、グラフ$K_{1, 3} \cup 2K_3 \cup 3K_1$ を示しています。

ここでも、$G_1$と$G_2$が頂点から離れたグラフの場合、
$G_1$と$G_2$の結合 ($G_1 + G_2$ と書く) は、$G_1 \cup G_2$の結合と、 $v_1v_2$ 型のすべての辺($v_1 \in V(G_1)$ $v_2 \in V(G_2)$)を組み合わせた
グラフであることを示します。 結合 $P_3 + K_2$ を図1-27に示します。

説明したいグラフの最終操作は、グラフ$G_1$ と $G_2$の場合より複雑です。
積$G_1 \times G_2$ には頂点集合 $V(G_1) \times V(G_2)$ と2つの頂点 $(u_1, u_2)$ と $(v_1, v_2)$があります。$u_1 = v_1$と$u_2v_2 \in E(G_2)$、
または $u_2 = v_2$ と $u_1v_1 \in E(G_1)$ のいずれかの場合にのみ、
$G_1 \times G_2$ で隣接している頂点の隣接性の定義の対称性は、
$G_1 \times G_2 \cong G_2 \times G_1$を (正しく) 示唆しています。
2つのグラフの積を視覚化する直感的な方法があります。$V(G_1) = {u_1, u_2, \dots, u_m}$
および $V(G_2) = {v_1, v_2, \dots, v_n}$ とします。 
$G_1$ の$n$個のコピー ($H_1, H_2, \dots, H_n$ と呼びます) を取り、$G_2$ の頂点 $v_1, v_2, \cdots, v_n$ の位置にそれぞれ配置します。
$v_jv_k \in E(G_2)$ の場合に限り、$H_j$ で $u_i$ とラベル付けされた頂点を$H_k$で$u_i$とラベル付けされた頂点に結合します。
この操作を図1-28 に示します。
\mysection{30ページ目翻訳}

非常によく知られているグラフのクラスは、製品の観点から説明することができます。
$n=1$ の場合、ハイパーキューブまたは$n$キューブ$Q_n$は$K_2$として定義され、
$$Q_n = Q_n-1 \times K_2$$ $$(n \geq 2 の場合)。$$
キューブ$Q_1, Q_2$、および $Q_3$ を図 1-29 に示します。
$n$-cubeを記述する別の方法は、$n$-tuple
の各項が0または1であるすべての$n$-tuple
のコレクションによってその頂点集合を表すことです。
ここで、2つの頂点$Q_n$は、対応する$n$-タプルは1つの座標だけが
異なります。図1-29は、
$n = 1, 2, 3$の場合の$Q_n$の頂点のラベル付けを示しています。
\mysection{31ページ目翻訳}

$n$-cubeを表現するもう一つの方法は、
その頂点集合をすべての$n$-tupleの集合で表すことである。
$n$-tupleの各項は0または1であり、$Q_n$
の二つの頂点は、対応する$n$-tupleが正確に一つの座標で
異なる場合にのみ隣接する。

図1-29は、$n=1, 2, 3$の場合の$Q_n$の頂点のラベリングである。

1.9 有向グラフ

ある状況を表現するのに、グラフが適切でない場合がある。
例えば、一方通行の道路がある道路地図は、グラフでは適切に表現することができない。
しかし、我々は「有向グラフ」を使うことができる。
有向グラフの定義の多くは、グラフの概念と密接に関連しているため、
ここでは簡潔かつ直感的に理解できるように説明する。

有向グラフ(またはダイグラフ)$D$は、頂点の有限で空でない集合$V(D)$と、
異なる頂点の順序付けられた組の(空の可能性のある)集合$E(D)$である。$E(D)$の要素は弧と呼ばれる。
\mysection{32ページ目翻訳}

グラフと同様に、ダイグラフはダイアグラムで表すことができます。
有向グラフ$D$の頂点は小さな円で示され、$D$の円弧$(u, v)$は、
頂点$u$から$v$に向かう曲線または線分を描くことによって表されます。
$(u, v)$と$(v, u)$は別個の円弧であり、
2つの頂点の方向が反対の場合、2つの円弧で結合できます。 
$V(D) = \{u, v, w, x\}$および $E(D) = \{(u,w), (v, w), (w, x), (x, w)\}$の有向グラフ$D$は、 
図1-30に示します。

有向グラフ$D$の基礎となるグラフは、すべてのアーク$(u, v)$または$(v, u)$を辺$uv$で置き換えることによって
$D$から得られるグラフ$G$です。 有向グラフ$D$とその下にあるグラフ$G$を図1-31に示します。

有向グラフ$D$の頂点の数はその順序と呼ばれ、$D$の弧の数はそのサイズです。$(u, v)$が$D$の弧である場合、$u$は$v$に隣接し、
$v$は$u$に隣接していると言われます。さらに、弧$(u, v)$は$u$から入射し、$v$に入射します。
有向グラフ$D$の頂点$v$の出次数od $v$ は$v$に隣接する頂点の数であり、$v$の内次数id $v$ は頂点の数 $\deg v = $od $v$ $+$ id $v$ です。 図 1-30 の有向グラフ$D$の頂点の
度数も図 1-30 に示されています。

有向グラフ理論の最初の定理は、グラフ理論の最初の定理に類似しています。

定理 1.7. 

$V(D) = \{v_1, v_2, \dots v_p\}$ で、次数が$p$でサイズが$q$の有向グラフを$D$とします。
したがって、$$ \sum_{i = 1}^{p} od v_i = \sum_{i = 1}^{p} id v_i = q$$
\mysection{33ページ目翻訳}

証拠。

D の頂点の出次数が合計されると、D の各弧は正確に 1 回カウントされます。 度数についても同じことが言えます。
同形の有向グラフには、期待される定義があります。 2 つの有向グラフ D1 と D2 は、V(D1) から V(D2) への 
1 対 1 の関数 fai (同型) が存在し、(u, v) が D1 のアークである場合、i (faiu , faiv) は D2 の円弧です。 
非形式的には、D1 と D2 は、どちらか一方を描画して他方を得ることができる場合、同形です。
サブダイグラフと誘導サブダイグラフは、グラフィカルな対応物と同じ方法で定義されます。 図 1-32 の有向グラフでは、F と H は D のサブ有向グラフです。 
F は D の誘導サブディグラフであり、H はそうではありません。
有向グラフ D のウォークは交互シーケンスです
W: v0、e1、v1、e2、v2、...、vn-1、en、vn (n>=0)
i = 1, 2, ..., n に対して ei = (vi-1, vi) となるように、頂点で始まり、頂点で終わる頂点と弧。 
このウォーク W は v0 - vn ウォークで、長さは n です。 トレイル、パス、サイクル、および回路の概念のように、有向グラフの場合は、
グラフの場合と同様に定義されますが、有向グラフでは常に円弧の方向に進みます。 有向グラフでは長さ 2 のサイクルが可能であることに注意してください。 
有向グラフ D のセミウォークは、交互のシーケンス W: v0、e1、v1、e2、v2、...、vn-1、en、vn (n >= 0) の頂点と弧であり、
ei = (vi -1, vi) または ei = (vi, vi-1) 各 i (1 <= i <= n). セミウォーク W は、長さ n の v0-vn セミウォークです。 
図 1-33 の有向グラフ D の場合、W: v, (v, w), w, (u, w), u, (x, u), x は、v-x ウォークではない v-x セミウォークです。 
実際、D には v-x ウォークが含まれていません。 セミウォークは弧の方向を無視するため、基礎となるグラフのウォークに対応します。
有向グラフ D の 2 つの頂点 u と v は、D に u-v セミウォークが含まれている場合に接続されます。 D の 2 つの頂点がすべて接続されている場合、
有向グラフ D は接続されています。つまり、基になるグラフが接続されている場合、D は接続されています。 図 1-33 の有向グラフ D を接続します。
\mysection{34ページ目翻訳}

切断された有向グラフと切断された有向グラフの要素は、
グラフと同じ方法で定義されます。有向グラフの場合、複数のタイプの接続性があります。
接続されている有向グラフは、弱接続と呼ばれることもあります。
有向グラフ$D$は、$D$の2 つの異なる頂点$u$と$v$ごとに、
$u-v$基本道または$v-u$基本道、またはその両方がある場合、
片側 (または片側接続) です。$D$の2つの異なる頂点$u$と$v$ごとに、
$u-v$基本道と$v-u$基本道がある場合、
有向グラフは強い (または強く接続されている) ことになります。
したがって、強い有向グラフは一方的であり、一方的な有向グラフは弱く接続されていますが、
逆の状態はどちらも真ではありません。 図1-33の有向グラフ$D$は
一方的ではありません (もちろん、強くもありません)。 図1-34の有向グラフ$D_1$は一方的ですが強くはありませんが、
$D2$は強いです。

$D$のすべての頂点$v$に対して$odv = id$ $v = r$ となる非負の整数$r$が存在する場合、
有向グラフ$D$は正則です。このような有向グラフは、$r$-正則とも呼ばれます。
$(u, v)$ が$D$の$n$アークである場合はいつでも、$(v, u)$ も同様である場合、
有向グラフ$D$を対称と呼びます。$D$が対称有向グラフの場合、$G$の各辺$uv$を弧$(u, v)$および$(v, u)$で置き換えることにより、
グラフ$G$から$D$を取得でき、$D = G^\ast$と記述します。 対称有向グラフ$K^\ast_{1,3}、P_4^\ast$および、$K_3^\ast$を図1-35に示します。


\mysection{35ページ目翻訳}

反対に、$(u, v)$ が$D$の弧であり、$(v, u)$が$D$の弧でない場合、
有向グラフ$D$は非対称であると言われます。 
2つの頂点は、多くても1つの円弧で結合されます。
2つの頂点がちょうど1つの円弧で結合されている有向グラフは、
トーナメントと呼ばれます。 これらの有向グラフについては、
第11章で詳しく説明します。図1-36は、2つの非対称の有向グラフを示しています。
$D_1$はトーナメントですが、$D_2$はそうではありません。


最後に、有向グラフ内の平行な弧が許可されている場合、
複数の有向グラフが得られます。 さらに、(有向) ループが許可されている場合は、
疑似有向グラフが生成されます。 多重有向グラフ$D_3$と疑似有向グラフ$D_4$を
図1-37に示します。
% \input{36.tex}
% \input{37.tex}
\end{document}