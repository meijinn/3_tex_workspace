\mysection{25ページ目翻訳}

1.8 特殊グラフ

グラフのあるクラスは頻繁に参照されるため、
特別な名前がつけられている。
まず、すべての可能な辺を持つグラフを考えることから始める。
すべての異なる2つの頂点が隣接する次数$p$のグラフは完全グラフであり、
Kpと表記される。
したがって、次数pの完全グラフは$(p - 1)$正則であり、
サイズ$q = (p/2) = p(p - 1)/2 $を持つ。
グラフ$K_p(1\leq p \leq 6)$を図1-21に示す。

1.6節で、パスとサイクルの概念を紹介しました。
順序$n \geq 1$のグラフでそれ自体がパスであるものを
順序$n$のパスと呼び、$P_n$と表記し、順序$n \geq 3$のグラフでそれ自体が
サイクルであるものを$C_n$と表記し、$n$サイクルと呼ぶ。
図1-22は3つのパスと3つのサイクルを示しています。
パスはその順序が偶数か奇数かによって長さが奇数か偶数になり、
サイクルはその長さが奇数か偶数かによって長さが偶数になる。
グラフ$G$は、$V(G)$が2つの(空でない)グラフに分割できるとき、二部構成である。
の頂点と$V$の頂点を結ぶ$G$のすべての辺が、$V$と$V$の部分集合になるようにする。