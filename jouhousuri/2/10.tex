\mysection{10ページ目翻訳}

$V(G_1)=V(G_2)$で、$E(G_1)=E(G_2)$のとき、2つのグラフ$G_1$と$G_2$は等しい。
確かに、等しいグラフは同型である。しかし、その逆は、グラフの順序が同じでなければならず、


しかし、図1-9のグラフ$G1、G2$は同型であるが等しくないので、逆は成り立たない。
(2つのグラフが等しくなるには、まず頂点集合が同じである必要がある。)

$G_1$と$G_2$が同型のグラフであるならば、同じ順序と同じ大きさを持つはずである。
なぜそうなのかを知るために、$V(G_1)$から$V(G_2)$への同型を$\phi$とする。
$\phi$は$V(G_1)$から$V(G_2)$への一対一の関数であるから、$G_1$と$G_2$が同じ順序を持つように$G_1$と$G_2$の頂点の対が存在することになる。
$u'$と$v'$を$G_2$の任意の2つの異なる頂点とする。
すると、$G_1$の頂点$u$と$v$には、$\phi u = u'、\phi v = v'$となるような、異なる頂点が存在する。
$u'$と$v'$が$G_2$で隣接するのは、$u$と$v$が$v$に隣接する場合のみであるから、$G_2$の$n$個の頂点$\phi v1, \phi v2, \ldots, \phi vn$は$G_1$で隣接し、
グラフ$G_1$と$G_2$は同じサイズである。

さらに、$G_1$の各頂点$v$について、$\deg_{G_1} v = \deg_{G_2} \phi v$となり、
これは以下のように検証することができる。
$deg_{G1} v = n$であり、$v_1, v_2, \ldots, v_n$が存在するとする。
$G_2$の$v_1, v_2, \ldots, \phi v_n$は$\phi v$に隣接し、$G_2$の他の頂点は$\phi v$に隣接していない。

それから、$n$個の頂点$\phi v, \phi v2, \phi vn$は$\phi v$に隣接し、$G_2$の他の頂点は$\phi v$に隣接しない。
したがって、$deg_{G2} \phi v = n$ したがって、2つのグラフが同型である場合、それらは必ず同じ順序と大きさを持ち、頂点の度数は同じである。(この
しかし、2つのグラフが等しいと言うことは、驚くことではありません。
同型は、本質的には同じであるという事実を表現するための形式的な方法に過ぎません。


は同じグラフになる)。しかし、これらの性質は、2つのグラフが次のようになるには十分ではありません。
は、非同型のグラフが同じ度数を持つことがあるので、同型であることがわかる。例として
図1-9のグラフ$G2, G3$は、次数5、サイズ6、度数3である。
3、3、2、2、2、でも$G_2\ncong G_3$。$G_2$、$G_3$を納得させる一つの方法として
同型でないことを示すには、$V(G_2)$から$V(G_3)$への1対1関数aがないことを示す必要がある。
は同型である可能性がある。このような関数aの場合、以下の3つが必要です。
$G2$の頂点で、$w_1, w_2, w_3$をイメージ頂点とするもの。
そのうちの2つはすべての頂点$w_1、w_2、w_3$は、$G_3$において隣接している。
$a$の下でそれらと共振する$G_2$もまた、隣接していなければならない。ただし、$G_2$ は
はそのような頂点を3つ含み、$a$は同型でない。したがって、同型は存在しない。
は、$V(G_2)$から$V(G_3)$への同型であり、$G2\ncong G3$である。
$G_1、G_2$は同型であるから、
上記の議論を$G_1$と$G_3$で繰り返すと、$G_1 \ncong  G_3$が示される。
2つのグラフが同じであるのは同型である場合だけなので、次数1のグラフは1つしかなく、これを単純グラフと呼ぶことにします。
図1-10に示すように、次数2のグラフは2つ(非同型)、次数3のグラフは4つあります。さらに、次数4のグラフは11個あります(問題2参照)。