\mysection{38ページ目翻訳}

我々はアルゴリズムという用語を、有限のステップ数で特定の入力から特定の出力を得るための、
よく定義された規則または命令のセットを意味するために使用します。
例えば、おなじみの正の整数の割り算で行われる一連の手順がアルゴリズムである。
数学におけるアルゴリズムの利用や関心は、主にコンピュータの普及に刺激され、
ここ数年で急激に高まっている。これから述べるように、アルゴリズムはグラフ理論の研究において
重要な役割を担っている。いくつかの基本的なアルゴリズムを調べる前に、
まず、アルゴリズムの中心的な話題である「複雑さ」について説明します。

アルゴリズムの複雑さは、コンピュータがそのアルゴリズムを使って問題を解くときに費やされる計算量の大きさを測るものである。
この尺度は、計算ステップの数、実行時間、または記憶領域を指すことがあります。
しかし、これら最初の2つには実質的な違いはなく、計算ステップ数が複雑さの解釈として使用されることになる。
アルゴリズムの複雑さは、一般に、入力データのサイズとプレゼンテーションの関数である。

ある問題を解決するためのアルゴリズムは1つだけではないことが多い。
例えば、一般的な計算問題は、(非特異)行列の逆行列を求めることである。
アルゴリズム$A$は$n \times n$行列の逆行列を$0.5n^4$単位の時間で求めるとします。
別のアルゴリズム$B$は、$2.5n^3$個の時間単位で$n \times n$個の行列の逆行列を求めます。
したがって、アルゴリズム$A$は$4 \times 4$ 行列の逆行列を128時間単位で求めるが、
アルゴリズム$B$は25\verb|%|長くかかる。実際、おなじみのガウス消去法を用いれば、
$n \times n$行列の逆行列を最大でも $cn^3$ 単位の時間で求めることができる(ここで$c$は定数)。
アルゴリズム$A$が$n\times n$行列の逆行列を求めるのに$0.5n^4$時間単位かかると言うのは、多くても$0.5n^4$時間単位という意味である。
実際には、比較的少数の$n\times n$行列が必要とする。$.5n^4$回単位です。