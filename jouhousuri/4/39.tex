\mysection{39ページ目翻訳}

つまり、最悪の場合にのみ、その時間が必要となる。
このような複雑さの指標をワーストケースの複雑さと呼ぶ。
対角行列の逆行列を求める場合など、最悪の場合の複雑さよりはるかに少ない時間で
済む行列の逆行列問題の例があることは間違いない。ワーストケース複雑度は最も一般的な複雑度の尺度であるが、
他の尺度もある。例えば、平均ケース複雑度は、すべての$n\times n$個の行列に対するアルゴリズムの平均実行時間を表します。
アルゴリズムは、その複雑さが入力サイズ(例えば$n$)の多項式である場合、効率的または「高速」である。
例えば、その複雑さがデータの入力サイズ(例えば$n$)の多項式であるか、
または$n$の多項式で境界付けられている場合、アルゴリズムは効率的または「高速」である、
のアルゴリズムが効率的であり、$2^n$, $n!$, $n^n$ のアルゴリズムは効率的ではありません。
計算問題は、その問題を解くための効率的なアルゴリズムが存在する場合、扱いやすいと呼ばれる。
計算問題は、その問題を解くための効率的なアルゴリズムが存在しないことが証明されれば、難解である。
本文中では、扱いやすい問題の例を見ていくことにする。しかし,多くの問題では,扱いやすいかどうかが分からない.
(この問題については2.4節で説明する)節で述べる)。
扱いにくい問題の例として、いわゆるタイリング問題(Wilf 9参照)がある。ある多角形に対して、
平面全体を多角形の形の床タイルで敷き詰めることができるか?図2-1は、長方形、正六角形、ひし形、および他の2つの多角形を示している。

図2-2は、図2-1の長方形を使った平面のタイル貼りと、正六角形を使ったタイル貼りを示しています。