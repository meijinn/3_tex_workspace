\mysection{21ページ目翻訳}

閉路とは、$n>3$, $v_o=v_n$, $n$個の頂点$v_1, v_2,\dots, v_n$
が別々である歩行$v_o, v_1, \cdots, v_n$のことである。
長さ$n$の閉路は$n$サイクルと呼ばれる。$u-v$の歩道は、
$u=v$なら閉歩道、$u\neq v$なら開歩道と言う。 
従って、閉路は閉歩道になり、自明でない道は開歩道になる。

自明でない閉歩道のことを回路という。
したがって,すべての閉路は回路であるが,
回路は閉路である必要はない。
(回路と閉路に関する定理1.3の類推は問題9を参照。)
図1-17のグラフに戻り、$u, w, y, v, u$は閉路(4サイクル)であり、
$u, x, w, v, y, w, u$は閉路でない回路であることに注意せよ。

グラフ$G$の頂点を$u$と$v$とし、
$G$に$u-v$の経路が含まれる場合、$u$は$v$に接続されているという。
グラフ$G$の頂点のすべての組$u,v$に対して$u$が$v$に接続されている場合、
グラフ$G$自体は接続されていることになる。
したがって、グラフ$G$は、$u-v$のパスが存在しない2つの頂点$u$と$y$が
存在する場合に切断される。このとき、自明グラフは連結されている。
グラフ$G$の部分グラフ$H$は、$H$が$G$の最大連結部分グラフであるとき、
$G$の構成要素である。

~に接続されている という関係は、グラフ$G$の頂点の集合上の同値関係である。
(問題2参照)とであり、したがって、$V(G)$の分割$V_1、V_2,\dots, V_k$を生じさせる。
グラフ$G$の構成要素の数を$k(G)$で表すと、
$G$の$1\leq i\leq k$の部分グラフ$\left\langle V_i \right\rangle _G$が$G$の構成要素である。
したがって、グラフ$G$は$k(G)=1$である場合にのみ連結される。図1-18のグラフ$G_1$は連結しているが、
$G_2$は$k(G_2)=4$で切断されている。