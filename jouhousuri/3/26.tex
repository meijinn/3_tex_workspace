\mysection{26ページ目翻訳}
図 1-23a のグラフは、$V(G)$ が部分集合 $V_1 = \{ v_1, v_6\} $ と $V_2 = \{ v_2, v_3, v_4, v_5\} $に分割され、
$G$の各辺がこれらの集合の頂点を結ぶため、2部グラフになります。このグラフは図 1-23b に再描画されており、
$V_1$ が上部に、$V_2$ が下部にあり、$G$ が2部構成である理由をより明確に示しています。
二部グラフとサイクルの間には特別な関係があります。

証明

まず、$G$ が2部グラフであると仮定します。 次に、$V(G)$ は、$G$の辺が$V_1$の頂点と$V_2$の頂点に結合するように、
2つの空でない部分集合$V_1$と$V_2$に分割できます。$G$にその$n$サイクル $C: v_1, v_2, \dots, v_n, v_1$が含まれているとします。 $n$が偶数であることを示します。 
$v_1 \in V_1$ とします。 辺$v_1v_2$は$V_1$の頂点と $V_2$ の頂点を結合するため、必然的に $v_2 \in V_2$です。
同じ理由で、$v_3 \in V_1, v_4 \in V_2$ などです。 $v_nv_1$ は $G$ と $v_1 \in V_1$ の辺であるため、$v_n \in V_2$ に従い、$n$は偶数になります。
逆に、$G$が奇数サイクルのないグラフであるとします。 私たちの目標は、$G$が二部であることを示すことです。 最初に$G$が連結されていると仮定し、
$v_1$を$G$の頂点とします。$G$は連結されているため、$G$の各頂点$v$に対して$G$に$v_1-v$基本道が存在します。