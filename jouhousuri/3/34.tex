\mysection{34ページ目翻訳}

切断された有向グラフと切断された有向グラフの要素は、
グラフと同じ方法で定義されます。有向グラフの場合、複数のタイプの接続性があります。
接続されている有向グラフは、弱接続と呼ばれることもあります。
有向グラフ$D$は、$D$の2 つの異なる頂点$u$と$v$ごとに、
$u-v$基本道または$v-u$基本道、またはその両方がある場合、
片側 (または片側接続) です。$D$の2つの異なる頂点$u$と$v$ごとに、
$u-v$基本道と$v-u$基本道がある場合、
有向グラフは強い (または強く接続されている) ことになります。
したがって、強い有向グラフは一方的であり、一方的な有向グラフは弱く接続されていますが、
逆の状態はどちらも真ではありません。 図1-33の有向グラフ$D$は
一方的ではありません (もちろん、強くもありません)。 図1-34の有向グラフ$D_1$は一方的ですが強くはありませんが、
$D2$は強いです。

$D$のすべての頂点$v$に対して$odv = id$ $v = r$ となる非負の整数$r$が存在する場合、
有向グラフ$D$は正則です。このような有向グラフは、$r$-正則とも呼ばれます。
$(u, v)$ が$D$の$n$アークである場合はいつでも、$(v, u)$ も同様である場合、
有向グラフ$D$を対称と呼びます。$D$が対称有向グラフの場合、$G$の各辺$uv$を弧$(u, v)$および$(v, u)$で置き換えることにより、
グラフ$G$から$D$を取得でき、$D = G^\ast$と記述します。 対称有向グラフ$K^\ast_{1,3}、P_4^\ast$および、$K_3^\ast$を図1-35に示します。

