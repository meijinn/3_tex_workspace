\mysection{52ページ目翻訳}

たとえば、句 $\{x_1, \overline{x}_3\}、\{x_1, \overline{x}_3\}、\{\overline{x}_1, x_3\}$ は満たされます。 
これを確認するには、$x_1=x_2=x_3=T$ とします。 ただし、句 $\{x_1, \overline{x}_2, \overline{x}_3\}、\{x_1, x_2\}、\{\overline{x}_1, x_2\}、\{\overline{x}_2, \overline{x}_1\}$、
および $\{x_3\}$ は使用できません。 満足できる。 そうであれば、$x_3 = T$ となります。 
さらに、$\{x_1, x_2\}$ と $\{\overline{x}_1, x_2\}$ は両方とも文節であり、$x_1$ と $\overline{x}_1$ は異なる真理値を持っているため、
$x_2 = T$、つまり $\overline{x}_2 = F$。 $\{\overline{x}_2, \overline{x}_1\}$ は句であるため、$x_1 = F$ になります。 ただし、$\{x_1, \overline{x}_2, \overline{x}_3\}$ は満たされません。

SAT で示される充足可能性問題では、次のように述べられています。
条項のセットが与えられた場合、それらは充足可能ですか? 
したがって、充足可能性の問題は決定問題です。 SAT の適切なエンコード スキームは、SAT のインスタンス $I$ を、$I$ 内のリテラルと
句の数の多項式となる文字列で記述するものです。

値 $x_1、x_2、\dots、x_n$ への真理値の割り当ては $2^n$ 通りあることに注意してください。 
これらすべての可能性を試すことで、SAT を解決できます。 ただし、文節の数が $n$ の多項式である場合、
このアルゴリズムは指数関数的に複雑になるため、(このような場合には) 効率的ではありません。 
$I$ が SAT のインスタンス、つまり節のセットであり、$C(I)$ が変数への真理値の代入である場合、
$C(I)$ が $I$ の各節が次のようになるかどうかを効率的にチェックできます。 満足。 
各節が満たされるように、真理値 $C(I)$ が $I$ の変数に割り当てられる場合に限り、$I$ は充足可能であるため、
SAT は NP に属します。 ただし、SAT が NP 完全問題であることを示すのは非常に複雑です。
したがって、この結果を証拠なしで述べます。

定理2.1。 SAT は NP 完全です。

定理 2.1 で使用されているテクニックのいくつかを説明するために、例を考えてみましょう。
$\pi$ を決定問題とします: 与えられたグラフ $G$ について、$V_i$ の 2 つの頂点が隣接しないように $V(G)$ を
3つのセット $V_1、V_2、V_3 $ に分割することは可能ですか $G$ では、$1 \leq i \leq 3$? $G$ を
図 2-4 に示す $\pi$ のインスタンスとします。

ここで、この $\pi$ のインスタンスを SAT のインスタンスに変換します。
$x_{i,j}$ をステートメントとします: 頂点 $i$ は $V_j (1 \leq i \leq 6 および 1 \leq j \leq 3)$ に属します。
各 $i (1 \leq i \leq 6)$ に対して、次の句を構築します。

$$A(i) = \{x_{i,1}, x_{i,2}, x_{i,3}\}$$
$$B(i) = \{\overline{x}_{i,1},\overline{x}_{i,2}\}$$
$$C(i) = \{\overline{x}_{i,1}, \overline{x}_{1,3}\}$$
$$D(i) = \{\overline{x}_{i,2}, \overline{x}_{i,3}\}$$