\mysection{40ページ目翻訳}

タイリング問題は難解であることがBerger [2]によって証明されている。
つまり、与えられたポリゴンで平面をタイリングできるかどうかを決定するための効率的な
アルゴリズムは存在しないのである。もちろん、この問題のいくつかの例については、
効率的なアルゴリズムが存在する。実際、タイリング問題は、長方形と正六角形については簡単に解ける。
また、正五角形についても簡単に解くことができます。正五角形では平面をタイル状にすることはできません
(問題1参照)。アルゴリズムの複雑さをより効果的に比較するために、次のように説明します。
関数の「次数」を説明する。fとgを正整数の集合で定義された2つの関数とする。
正整数の集合に定義された2つの関数があるとする。fの次数はgの次数より低いか等しいとする。
が存在するとき、fの次数はgの次数以下であるという。
が存在する。
すべての$n>n_o$に対して。$f$の次数が$g$の次数以下である場合、$f(n) = C(g(n))$ と書いたり、
$f(n)$ は$O(g(n))$ $(f(n)$ は ”big oh” の $g(n)$ と読みます)と言ったりします。
これは、$f$が$g$より速く成長しないことを意味します。
関数$f$は$g$よりゆっくり成長するか、同じ速度で成長するかです。
関数$f$と$g$は、$f(n)=O(g(n) と g(n) = Of(n))$であれば同じ次数である。
$f(n)=Cg(n)$である。
$f(n)=3n+5$, $g(n)=n^2$であるとする。すると、$f(n)=O(g(n))$となる。
定数$C=1$に対して
すべての$n>4 = n_o$に対して$3n+5 \leq 1n^2$となる。この不等式は数学的帰納法で検証することができるが、
別の方法を用いる。$n>5$のとき$3/n \leq 3/5$, $n \leq 5$のとき$5/n^2 \leq 1/5$であることに注目する。
つまり、$3/n+5/n^2\leq 3/5 + 1/5 < 1$は$n>4$のとき、
$n^2$ に $(3/n + 5/n^2) < 1$ を掛けると、目的の結果が得られる。
また、$C$と$n_0$を他の値にすることで、$f(n) = O(g(n))$ を検証することができた。
例えば、$C = 3$とする。すると、$n>2$に対して$3n +5 \leq 3n^2$となることがわかります。

例えば、$C =11$、$n_0 = 0$。$f(n) = .5n^4$, $g(n) = 2.5n^2$ と仮定する。
すると、$g(n)=Of(n)$となり、$g(n)はO(n)$となる。また、$f(n)はO(n)$であるが、$f(n)$は$O(g(n))$である。
$h(n)$ $i$が次数$k$の任意の$n$の多項式であれば、$h(n) = 0(n*)$となる。アルゴリズム$A$と$B$がそれぞれ複雑度関数$f(n)$と$g(n)$
を持つ場合、$f(n)=O(g(n))$、$g(n)=O(f(n))$のとき、アルゴリズム$B$より$1s$効率が高いと言う。関数の増加順序の階層を知るために、
7つの一般的な順序を列挙する。$a,b>1$であれば$\log n$は$\log n$の定数倍であるから、
対数の底は重要ではない。

増加する順序のヒエラルキー:

すべての$n \leq 1$に対して$\log n < n$なので、$log n$は$O(n^2), O(n^3), O(2^n)$ です。