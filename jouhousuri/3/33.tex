\mysection{33ページ目翻訳}

証拠。

D の頂点の出次数が合計されると、D の各弧は正確に 1 回カウントされます。 度数についても同じことが言えます。
同形の有向グラフには、期待される定義があります。 2 つの有向グラフ D1 と D2 は、V(D1) から V(D2) への 
1 対 1 の関数 fai (同型) が存在し、(u, v) が D1 のアークである場合、i (faiu , faiv) は D2 の円弧です。 
非形式的には、D1 と D2 は、どちらか一方を描画して他方を得ることができる場合、同形です。
サブダイグラフと誘導サブダイグラフは、グラフィカルな対応物と同じ方法で定義されます。 図 1-32 の有向グラフでは、F と H は D のサブ有向グラフです。 
F は D の誘導サブディグラフであり、H はそうではありません。
有向グラフ D のウォークは交互シーケンスです
W: v0、e1、v1、e2、v2、...、vn-1、en、vn (n>=0)
i = 1, 2, ..., n に対して ei = (vi-1, vi) となるように、頂点で始まり、頂点で終わる頂点と弧。 
このウォーク W は v0 - vn ウォークで、長さは n です。 トレイル、パス、サイクル、および回路の概念のように、有向グラフの場合は、
グラフの場合と同様に定義されますが、有向グラフでは常に円弧の方向に進みます。 有向グラフでは長さ 2 のサイクルが可能であることに注意してください。 
有向グラフ D のセミウォークは、交互のシーケンス W: v0、e1、v1、e2、v2、...、vn-1、en、vn (n >= 0) の頂点と弧であり、
ei = (vi -1, vi) または ei = (vi, vi-1) 各 i (1 <= i <= n). セミウォーク W は、長さ n の v0-vn セミウォークです。 
図 1-33 の有向グラフ D の場合、W: v, (v, w), w, (u, w), u, (x, u), x は、v-x ウォークではない v-x セミウォークです。 
実際、D には v-x ウォークが含まれていません。 セミウォークは弧の方向を無視するため、基礎となるグラフのウォークに対応します。
有向グラフ D の 2 つの頂点 u と v は、D に u-v セミウォークが含まれている場合に接続されます。 D の 2 つの頂点がすべて接続されている場合、
有向グラフ D は接続されています。つまり、基になるグラフが接続されている場合、D は接続されています。 図 1-33 の有向グラフ D を接続します。