\mysection{50ページ目翻訳}

なぜこれらの問題が研究に値するのか疑問に思う人もいるかもしれません。
この分野の多くの専門家は、すべての NP 完全問題は解決困難であると考えているため、
これらの問題の 1 つを解決するための効率的なアルゴリズムを見つける試みは優先順位が低いはずです。
ただし、問題が NP 完全であることがわかったからといって、問題を放棄する必要があるわけではありません。
むしろ、この知識が自分のアプローチの方向性を導くはずです。たとえば、問題の特定の特殊なケースを解決する効率的なアルゴリズムを見つけようとしたり、
条件を緩和して効率的に解決できる新しい問題を取得したりすることができます。最適化問題の場合、
常に「最適に近い」解を保証する効率的なアルゴリズムの開発を試みることができます。採用されているいくつかのアプローチについては、後続の章で説明します。

ここで、NP 完全問題について非公式に紹介します。正式なアプローチでは、
計算モデルとして「チューリング マシン」を使用する必要があります。
これは、Garey と Johnson の書籍に記載されています [7.第2章]、イーブン[6]、ウィルフ[9]。 
NP 完全問題をより注意深く説明するには、次の用語が必要です。問題のインスタンスとは、問題のパラメーター (つまり、入力) を完全に記述する特定の値または数量のセットを意味します。
問題を記述するために使用される方法は、エンコード スキームとも呼ばれます。グラフに適用される問題の場合、適切なエンコード スキームは通常、
長さが次の多項式である文字列によって問題のインスタンスを記述するものです。

順序とサイズ。たとえば、グラフ同型問題は、与えられた 2 つのグラフ G と G2 が同型であるかどうかを判断することです。
この問題の例は、G1 と G2 の頂点とエッジのセットを指定することによって与えられます。これは合理的なエンコード スキームです。

問題のすべてのインスタンスの出力が「はい」または「いいえ」のいずれかであるという特性を持つ問題は、
決定問題と呼ばれます。の「はいインスタンス」または「いいえインスタンス」は、出力がそれぞれ「はい」または「いいえ」になる
インスタンスです。したがって、決定問題 は、D- で示される問題  のすべてのインスタンスと、すべての Yes インスタンスのサブセット Y、CD で構成されます。

たとえば、問題: 「G G は (2 つの与えられたグラフ G1 と
G2)?" は決定問題です。この例では、Y はすべてのペア G1、G2 で構成されます。

G1 = G2 のようなグラフ - 意思決定問題 の非決定的アルゴリズムには 2 つの段階があります。 の特定のインスタンス I に対して、
最初の段階では、証明書 C(I) とも呼ばれる構造 (または潜在的な解決策) を推測します。第 2 段階では、入力 I と C(I) を使用して、
I が Y にあるかどうかを C(I) で判断できるかどうかを効率的にチェックします。直感的に、証明書を推測することは、
答えを推測することと同じです。たとえば、π をグラフ同型問題 (決定問題として記述) とします。 G1 と G2 が同じ次数列を持つ
 2 つのグラフであると仮定します。 の証明書の 1 つの可能性は、V(G) から V(G2) への 1 対 1 関数です。このような証明書が推測されると、 
 if uv E E(G) の場合および du v E E(G2) の場合に限り、それが同型写像であるかどうかを効率的にチェックできます。この場合、チェック段階の複雑さは O(g) 
 であることに注意してください。

証明書の選択には意味がある必要があります。特に、$\pi$ が決定問題である場合、D のすべてのインスタンス に対して、
次のことが当てはまります: (1) I EY の場合、その後