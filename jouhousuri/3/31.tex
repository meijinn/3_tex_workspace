\mysection{31ページ目翻訳}

$n$-cubeを表現するもう一つの方法は、
その頂点集合をすべての$n$-tupleの集合で表すことである。
$n$-tupleの各項は0または1であり、$Q_n$
の二つの頂点は、対応する$n$-tupleが正確に一つの座標で
異なる場合にのみ隣接する。

図1-29は、$n=1, 2, 3$の場合の$Q_n$の頂点のラベリングである。

1.9 有向グラフ

ある状況を表現するのに、グラフが適切でない場合がある。
例えば、一方通行の道路がある道路地図は、グラフでは適切に表現することができない。
しかし、我々は「有向グラフ」を使うことができる。
有向グラフの定義の多くは、グラフの概念と密接に関連しているため、
ここでは簡潔かつ直感的に理解できるように説明する。

有向グラフ(またはダイグラフ)$D$は、頂点の有限で空でない集合$V(D)$と、
異なる頂点の順序付けられた組の(空の可能性のある)集合$E(D)$である。$E(D)$の要素は弧と呼ばれる。