\mysection{32ページ目翻訳}

グラフと同様に、ダイグラフはダイアグラムで表すことができます。
有向グラフ$D$の頂点は小さな円で示され、$D$の円弧$(u, v)$は、
頂点$u$から$v$に向かう曲線または線分を描くことによって表されます。
$(u, v)$と$(v, u)$は別個の円弧であり、
2つの頂点の方向が反対の場合、2つの円弧で結合できます。 
$V(D) = \{u, v, w, x\}$および $E(D) = \{(u,w), (v, w), (w, x), (x, w)\}$の有向グラフ$D$は、 
図1-30に示します。

有向グラフ$D$の基礎となるグラフは、すべてのアーク$(u, v)$または$(v, u)$を辺$uv$で置き換えることによって
$D$から得られるグラフ$G$です。 有向グラフ$D$とその下にあるグラフ$G$を図1-31に示します。

有向グラフ$D$の頂点の数はその順序と呼ばれ、$D$の弧の数はそのサイズです。$(u, v)$が$D$の弧である場合、$u$は$v$に隣接し、
$v$は$u$に隣接していると言われます。さらに、弧$(u, v)$は$u$から入射し、$v$に入射します。
有向グラフ$D$の頂点$v$の出次数od $v$ は$v$に隣接する頂点の数であり、$v$の内次数id $v$ は頂点の数 $\deg v = $od $v$ $+$ id $v$ です。 図 1-30 の有向グラフ$D$の頂点の
度数も図 1-30 に示されています。

有向グラフ理論の最初の定理は、グラフ理論の最初の定理に類似しています。

定理 1.7. 

$V(D) = \{v_1, v_2, \dots v_p\}$ で、次数が$p$でサイズが$q$の有向グラフを$D$とします。
したがって、$$ \sum_{i = 1}^{p} od v_i = \sum_{i = 1}^{p} id v_i = q$$