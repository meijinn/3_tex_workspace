\mysection{演習19}
\subsection{実行プログラム}
実行プログラムをソースコード\ref{s19}に示す。
パラメータ値は、\textgt{RADIUS}を変化させると、
回転量が変化し、ロボットの回転と進行距離の両方が代わるため、
DISTANCEの値のみを変更した。\textgt{DISTANCE}の実測値を計測し、
DISTANCEの値を上下に変化させて、34.0から34.05に変更した。
360度回転は、指示値通りに動くが、直進の際の始動にモータの左右差が
あったため、それを考慮した上で34.05の値とした。

\begin{lstlisting}[caption=演習19のプログラム,label=s19]
#include <stdio.h>
#include <math.h>
#include <process.h>
#include <dos.h>
#include "v25.h"
#include "ms.h"

#define D2R         (M_PI/180)
#define R2D         (180/M_PI)
#define RADIUS	15.0
#define DISTANCE    34.05
#define REDUCTION   19.225
#define CPR         400
#define MMPS        (60*REDUCTION/(2.0*M_PI*RADIUS))
#define DPC         (2.0*M_PI*RADIUS/((float)CPR*REDUCTION))
#define APC         (DPC/(2*DISTANCE))
#define DMAX        1000
#define THRESH_D    10
#define THRESH_Q    5*D2R
#define THRESH_V    50

#define ABS(x)      (x<0?-(x):(x))

void estimate(double *px,double *py,double *pq,long *pt);
void translate(double d,double vmax,double gain);
void rotate(double qd,double wmax,double gain);

int k=0;
int xi[DMAX],yi[DMAX],qi[DMAX];
long ti[DMAX];

int main(void)
{
    double x,y,q;
    int i;
    int kp=8,ki_inv=8;
    int vmax = 100;
    int wmax = 100/DISTANCE;
    int nl,nr;
    long t,nt;
    FILE *fp;

    ms_init();
    ms_motor_on();
    ms_set_gain(kp,ki_inv);

    estimate(&x,&y,&q,&t);
    rotate(360*D2R,wmax,5);
    translate(1000,vmax,5);

    ms_set_v(0,0);
    do{
        estimate(&x,&y,&q,&t);
        ms_read_v(&nl,&nr,&nt);
    }while(nl != 0 || nr != 0);

    ms_motor_off();

    if((fp=fopen("data.dat","wt"))==NULL)
    {
        printf("Can't open file...\n");
    }

    for(i=0;i<k;i++)
    {
        fprintf(fp,"%5d %5d %5d %5ld\n",xi[i],yi[i],qi[i],ti[i]);
    }
    fclose(fp);
    return 0;
}

void estimate(double *px,double *py,double *pq,long *pt)
{
    static long cl0=0,cr0=0;
    static double x=0.0,y=0.0,q=0.0;
    long cl,cr,ct;
    int dl,dr;
    double ds;

    ms_read_c(&cl,&cr,&ct);
    dl=cl-cl0;
    dr=cr-cr0;
    cl0=cl;
    cr0=cr;
    ds=DPC/2*(dr+dl);
    x += ds*cos(q);
    y += ds*sin(q);
    q += APC*(dr-dl);

    *px=x;
    *py=y;
    *pq=q;
    *pt=ct;
    if (k<DMAX)
    {
        xi[k]=10*x;
        yi[k]=10*y;
        qi[k]=1000*q;
        ti[k]=ct;
        k++;
    }    

}

void translate(double d,double vmax,double gain)
{
    double x0,y0,q0,x,y,q;
    long t0,t,nt;
    double cosq0,sinq0,dd,vc;
    int nc,nl,nr;

    estimate(&x0,&y0,&q0,&t0);
    cosq0=cos(q0);
    sinq0=sin(q0);
    x=x0;
    y=y0;
    q=q0;
    t=t0;

    do
    {
        dd=cosq0*(x-x0)+sinq0*(y-y0);
        vc=gain*(d-dd);
        if (ABS(vc)>vmax)
        {
            vc=(vc<0.0)?-vmax:vmax;
        }
        nc=vc*MMPS;
        ms_set_v(nc,nc);
        estimate(&x,&y,&q,&t);
        ms_read_v(&nl,&nr,&nt);
    } while (ABS(d-dd)>THRESH_D||ABS(nl)>THRESH_V||ABS(nr)>THRESH_V);
}

void rotate(double qd,double wmax,double gain)
{
  double x0,y0,q0,x,y,q;
  long t0,t,nt;
  double qq,wc,v,dw;
  int nl,nr;

  estimate(&x0,&y0,&q0,&t0);
  x=x0;
  y=y0;
  q=q0;
  t=t0;

  do{
  qq=q-q0;
  wc=gain*(qd-qq);
  if(ABS(wc)>wmax)
  {
    wc=(wc<0.0) ? -wmax:wmax;
  }
  dw=DISTANCE*wc;
  nl=-MMPS*dw;
  nr=MMPS*dw;
  ms_set_v(nl,nr);
  estimate(&x,&y,&q,&t);
  ms_read_v(&nl,&nr,&nt);
  }while (ABS(qd-qq)>THRESH_Q||ABS(nl)>THRESH_V||ABS(nr)>THRESH_V);
}
\end{lstlisting}

\mysubsection{実行結果}
ロボットが、その場で360度回転して1000[mm]直進する。
回転角度を360度より浅くして、直進動作時に左右のモータの始動差を吸収して、
360度方向に直進する。

\mysection{演習20}
\subsection{実行プログラム}
実行プログラムをソースコード\ref{s20}に示す。
ロボットの初期位置を$(x_s, y_s, \theta_s) = (0, 0, 0)$としている。

\begin{lstlisting}[caption=演習20のプログラム,label=s20]
#include <stdio.h>
#include <math.h>
#include <process.h>
#include <dos.h>
#include "v25.h"
#include "ms.h"

#define D2R         (M_PI/180)
#define R2D         (180/M_PI)
#define RADIUS	15.0
#define DISTANCE    34.05
#define REDUCTION   19.225
#define CPR         400
#define MMPS        (60*REDUCTION/(2.0*M_PI*RADIUS))
#define DPC         (2.0*M_PI*RADIUS/((float)CPR*REDUCTION))
#define APC         (DPC/(2*DISTANCE))
#define DMAX        1000
#define THRESH_D    10
#define THRESH_Q    5*D2R
#define THRESH_V    50

#define ABS(x)      (x<0?-(x):(x))

void estimate(double *px,double *py,double *pq,long *pt);
void translate(double d,double vmax,double gain);
void rotate(double qd,double wmax,double gain);

int k=0;
int xi[DMAX],yi[DMAX],qi[DMAX];
long ti[DMAX];

int main(void)
{
    double x,y,q;
    double x_d=0,x_s=0,y_d=0,y_s=0,th_s=0,th_d=0,th_1=0,th_2=0,a_0=0;
    int i;
    int kp=8,ki_inv=8;
    int vmax = 100;
    int wmax = 100/DISTANCE;
    int nl,nr;
    long t,nt;
    FILE *fp;




    while(1){
    ms_init();
        ms_motor_on();
        ms_set_gain(kp,ki_inv);
    printf("x_d y_d th_d\n");
    scanf("%lf %lf %lf",&x_d,&y_d,&th_d);
        estimate(&x,&y,&q,&t);
    th_1=atan2(y_d-y_s,x_d-x_s)-th_s;
        rotate(th_1,wmax,5);
    a_0=sqrt((x_d-x_s)*(x_d-x_s)+(y_d-y_s)*(y_d-y_s));
        translate(a_0,vmax,5);
    th_2=th_d*D2R-th_1-th_s;
    rotate(th_2,wmax,5);
    printf("aaa\n");
    x_s=x_d;
    y_s=y_d;
    th_s=th_d;

        ms_set_v(0,0);
        do{
        estimate(&x,&y,&q,&t);
        ms_read_v(&nl,&nr,&nt);
        }while(nl != 0 || nr != 0);

    ms_motor_off();
    }
    return 0;
}

void estimate(double *px,double *py,double *pq,long *pt)
{
    static long cl0=0,cr0=0;
    static double x=0.0,y=0.0,q=0.0;
    long cl,cr,ct;
    int dl,dr;
    double ds;

    ms_read_c(&cl,&cr,&ct);
    dl=cl-cl0;
    dr=cr-cr0;
    cl0=cl;
    cr0=cr;
    ds=DPC/2*(dr+dl);
    x += ds*cos(q);
    y += ds*sin(q);
    q += APC*(dr-dl);

    *px=x;
    *py=y;
    *pq=q;
    *pt=ct;
    if (k<DMAX)
    {
        xi[k]=10*x;
        yi[k]=10*y;
        qi[k]=1000*q;
        ti[k]=ct;
        k++;
    }    

}

void translate(double d,double vmax,double gain)
{
    double x0,y0,q0,x,y,q;
    long t0,t,nt;
    double cosq0,sinq0,dd,vc;
    int nc,nl,nr;

    estimate(&x0,&y0,&q0,&t0);
    cosq0=cos(q0);
    sinq0=sin(q0);
    x=x0;
    y=y0;
    q=q0;
    t=t0;

    do
    {
        dd=cosq0*(x-x0)+sinq0*(y-y0);
        vc=gain*(d-dd);
        if (ABS(vc)>vmax)
        {
            vc=(vc<0.0)?-vmax:vmax;
        }
        nc=vc*MMPS;
        ms_set_v(nc,nc);
        estimate(&x,&y,&q,&t);
        ms_read_v(&nl,&nr,&nt);
    } while (ABS(d-dd)>THRESH_D||ABS(nl)>THRESH_V||ABS(nr)>THRESH_V);
}

void rotate(double qd,double wmax,double gain)
{
    double x0,y0,q0,x,y,q;
    long t0,t,nt;
    double qq,wc,v,dw;
    int nl,nr;

    estimate(&x0,&y0,&q0,&t0);
    x=x0;
    y=y0;
    q=q0;
    t=t0;

    do{
    qq=q-q0;
    wc=gain*(qd-qq);
    if(ABS(wc)>wmax)
    {
        wc=(wc<0.0) ? -wmax:wmax;
    }
    dw=DISTANCE*wc;
    nl=-MMPS*dw;
    nr=MMPS*dw;
    ms_set_v(nl,nr);
    estimate(&x,&y,&q,&t);
    ms_read_v(&nl,&nr,&nt);
    printf("%lf %lf %lf\n", ABS(qd-qq), ABS(nl), ABS(nr));
    }while (ABS(qd-qq)>THRESH_Q||ABS(nl)>THRESH_V||ABS(nr)>THRESH_V);
}
\end{lstlisting}

\mysubsection{実行結果}
目標位置$(x_d, y_d, \theta_d)$を入力すると、
目標方向を向き、目標の座標$(x_d, y_d)$の方向に直進し、
到着すると$\theta_d$方向に向きを変える。
目標位置・姿勢の状態をとると、初期値を現在値に更新し、
再び目標位置の入力待機を行う。その後、目標位置の入力と移動の処理を繰り返して行うことができる。
その際、目標位置の入力前にモータの電圧をオンと初期化、出力後にモータの電圧をオフ
を毎ループごとに繰り返し行う。

\mysection{演習21}
\subsection{実行プログラム}
実行プログラムをソースコード\ref{s21}に示す。

\begin{lstlisting}[caption=演習21のプログラム,label=s21]
#include <stdio.h>
#include <math.h>
#include <process.h>
#include <dos.h>
#include "v25.h"
#include "ms.h"

#define D2R         (M_PI/180)
#define R2D         (180/M_PI)
#define RADIUS	15.0
#define DISTANCE    34.05
#define REDUCTION   19.225
#define CPR         400
#define MMPS        (60*REDUCTION/(2.0*M_PI*RADIUS))
#define DPC         (2.0*M_PI*RADIUS/((float)CPR*REDUCTION))
#define APC         (DPC/(2*DISTANCE))
#define DMAX        1000
#define THRESH_D    10
#define THRESH_Q    5*D2R
#define THRESH_V    1

#define GOALX       1200
#define GOALY       1000
#define GOALQ       0
#define VMAX        150
#define WMAX        (VMAX/(3*DISTANCE))
#define REVERSE     200

#define ABS(x)      (x<0?-(x):(x))

void estimate(double *px,double *py,double *pq,long *pt);
void translate(double d,double vmax,double gain);
int translate2(double d,double vmax,double gain);
void rotate(double qd,double wmax,double gain);
double seikika(double angle);
void avoid(int j);

int k=0;
int xi[DMAX],yi[DMAX],qi[DMAX];
long ti[DMAX];

int main(void)
{
    double x,y,q,q1,q2,dl,dq;
    int i,j;
    int kp=8,ki_inv=8;
    int nl,nr;
    long t,nt;
    FILE *fp;

    ms_init();
    ms_set_gain(kp,ki_inv);
    ms_motor_on();
    ms_wait(1000);

    while (1)
    {
        estimate(&x,&y,&q,&t);

        dl=sqrt((GOALX-x)*(GOALX-x)+(GOALY-y)*(GOALY-y));
        if(dl<THRESH_D) break;

        dq=atan2(GOALY-y,GOALX-x)-q;
        dq=seikika(dq);
        rotate(dq,WMAX,5);

        j=translate2(dl,VMAX,5);

        avoid(j);
    }

    while (1)
    {
        estimate(&x,&y,&q,&t);

        dq=seikika(GOALQ-q);
        if (ABS(dq)<THRESH_Q)   break;

        rotate(dq,WMAX,5);
    }

      ms_motor_off();

    if((fp=fopen("data.dat","wt"))==NULL)
    {
        printf("Can't open file...\n");
        exit(1);
    }

    for(i=0;i<k;i++)
    {
        fprintf(fp,"%5d %5d %5d %5ld\n",xi[i],yi[i],qi[i],ti[i]);
    }
    fclose(fp);
    return 0;
}

void estimate(double *px,double *py,double *pq,long *pt)
{
    static long cl0=0,cr0=0;
    static double x=0.0,y=0.0,q=0.0;
    long cl,cr,ct;
    int dl,dr;
    double ds;

    ms_read_c(&cl,&cr,&ct);
    dl=cl-cl0;
    dr=cr-cr0;
    cl0=cl;
    cr0=cr;
    ds=DPC/2*(dr+dl);
    x += ds*cos(q);
    y += ds*sin(q);
    q += APC*(dr-dl);

    *px=x;
    *py=y;
    *pq=q;
    *pt=ct;
    if (k<DMAX)
    {
        xi[k]=10*x;
        yi[k]=10*y;
        qi[k]=1000*q;
        ti[k]=ct;
        k++;
    }    

}

void translate(double d,double vmax,double gain)
{
    double x0,y0,q0,x,y,q;
    long t0,t,nt;
    double cosq0,sinq0,dd,vc;
    int nc,nl,nr;

    estimate(&x0,&y0,&q0,&t0);
    cosq0=cos(q0);
    sinq0=sin(q0);
    x=x0;
    y=y0;
    q=q0;
    t=t0;

    do
    {
        dd=cosq0*(x-x0)+sinq0*(y-y0);
        vc=gain*(d-dd);
        if (ABS(vc)>vmax)
        {
            vc=(vc<0.0)?-vmax:vmax;
        }
        nc=vc*MMPS;
        ms_set_v(nc,nc);
        estimate(&x,&y,&q,&t);
        ms_read_v(&nl,&nr,&nt);
    } while (ABS(d-dd)>THRESH_D||ABS(nl)>THRESH_V||ABS(nr)>THRESH_V);
}

int translate2(double d,double vmax,double gain)
{
    double x0,y0,q0,x,y,q,g;
    long t0,t,nt;
    double cosq0,sinq0,dd,vc;
    int nc,nl,nr,s[3],check;

    estimate(&x0,&y0,&q0,&t0);
    cosq0=cos(q0);
    sinq0=sin(q0);
    x=x0;
    y=y0;
    q=q0;
    t=t0;

    do
    {
        ms_ifr(s);
        check = 1*s[0]+2*s[1]+4*s[2];
        ms_read_v(&nl,&nr,&nt);

        if (check)
        {
            for (g = 1; g > 0; g-=0.01)
                ms_set_v((int)(nl*g),(int)(nr*g));
            
            return check;   
        }
        
        dd=cosq0*(x-x0)+sinq0*(y-y0);

        vc=gain*(d-dd);

        if (ABS(vc)>vmax)
        {
            vc = (vc < 0.0) ? -vmax : vmax;
        }
        nc=vc*MMPS;
        ms_set_v(nc,nc);
        estimate(&x,&y,&q,&t);
        ms_read_v(&nl,&nr,&nt);
    } while (ABS(d-dd)>THRESH_D||ABS(nl)>THRESH_V||ABS(nr)>THRESH_V);
    return 0;
}

void rotate(double qd,double wmax,double gain)
{
  double x0,y0,q0,x,y,q;
  long t0,t,nt;
  double qq,wc,v,dw;
  int nl,nr;

  estimate(&x0,&y0,&q0,&t0);
  x=x0;
  y=y0;
  q=q0;
  t=t0;

  do{
  qq=q-q0;
  wc=gain*(qd-qq);
  if(ABS(wc)>wmax)
  {
    wc=(wc<0.0) ? -wmax:wmax;
  }
  dw=DISTANCE*wc;
  nl=-MMPS*dw;
  nr=MMPS*dw;
  ms_set_v(nl,nr);
  estimate(&x,&y,&q,&t);
  ms_read_v(&nl,&nr,&nt);
  }while (ABS(qd-qq)>THRESH_Q||ABS(nl)>THRESH_V||ABS(nr)>THRESH_V);
}

double seikika(double angle)
{
    while(angle>M_PI)   angle -= 2*M_PI;
    while(angle<=-M_PI)   angle += 2*M_PI;
    return angle;
}

void avoid(int j)
{
    if(j!=0)
    {
        translate(-REVERSE,VMAX,5);
        if(j&1) rotate(-60*D2R,WMAX,5);
        else rotate(60*D2R,WMAX,5);
        translate(REVERSE,VMAX,5);
    }
}
\end{lstlisting}

\mysubsection{実行結果}
ロボットを起動させると、前進し、近接センサが障害物に反応すると、
ロボットは後退し、一旦後退する。その後センサ値が1であれば60度右に旋回し、
それ以外ならば、左に60度旋回する。その後、前方に進む。


\mysection{演習22}
\subsection{実行プログラム}
実行プログラムをソースコード\ref{s22}に示す。
本プログラムは、移動平均法を用いて、近接センサの値を平均化し、
標本化している。過去5つのセンサデータ(0か1)を保持し、その平均値が0.8
以上なら障害物を検知した場合の処理を行う。

\begin{lstlisting}[caption=演習22のプログラム,label=s22]
#include <stdio.h>
#include <math.h>
#include <process.h>
#include <dos.h>
#include "v25.h"
#include "ms.h"

#define D2R         (M_PI/180)
#define R2D         (180/M_PI)
#define RADIUS	15.0
#define DISTANCE    34.05
#define REDUCTION   19.225
#define CPR         400
#define MMPS        (60*REDUCTION/(2.0*M_PI*RADIUS))
#define DPC         (2.0*M_PI*RADIUS/((float)CPR*REDUCTION))
#define APC         (DPC/(2*DISTANCE))
#define DMAX        1000
#define THRESH_D    10
#define THRESH_Q    5*D2R
#define THRESH_V    1

#define GOALX       1200
#define GOALY       1000
#define GOALQ       0
#define VMAX        150
#define WMAX        (VMAX/(3*DISTANCE))
#define REVERSE     200

#define ABS(x)      (x<0?-(x):(x))

#define WIDTH 10    /*センサのデータを保存する個数*/

void estimate(double *px,double *py,double *pq,long *pt);
void translate(double d,double vmax,double gain);
int translate2(double d,double vmax,double gain);
void rotate(double qd,double wmax,double gain);
double seikika(double angle);
void avoid(int j);

int k=0;
int xi[DMAX],yi[DMAX],qi[DMAX];
long ti[DMAX];

    int sum_s [3]={0,0,0};   /*直近10個のデータの合計    0:左,1:中央,2:右(bs)*/
    int bp=0;   /*最新データの場所*/
    int data_s[3][WIDTH];  /*直近10個のデータ (buff)*/

int main(void)
{
    double x,y,q,q1,q2,dl,dq;
    int i,j;
    int kp=8,ki_inv=8;
    int nl,nr;
    long t,nt;
    FILE *fp;

    ms_init();
    ms_set_gain(kp,ki_inv);
    ms_motor_on();
    ms_wait(1000);

    while (1)
    {
        estimate(&x,&y,&q,&t);

        dl=sqrt((GOALX-x)*(GOALX-x)+(GOALY-y)*(GOALY-y));
        if(dl<THRESH_D) break;

        dq=atan2(GOALY-y,GOALX-x)-q;
        dq=seikika(dq);
        rotate(dq,WMAX,5);

        j=translate2(dl,VMAX,5);

        avoid(j);
    }

    while (1)
    {
        estimate(&x,&y,&q,&t);

        dq=seikika(GOALQ-q);
        if (ABS(dq)<THRESH_Q)   break;

        rotate(dq,WMAX,5);
    }

      ms_motor_off();

    if((fp=fopen("data.dat","wt"))==NULL)
    {
        printf("Can't open file...\n");
        exit(1);
    }

    for(i=0;i<k;i++)
    {
        fprintf(fp,"%5d %5d %5d %5ld\n",xi[i],yi[i],qi[i],ti[i]);
    }
    fclose(fp);
    return 0;
}

void estimate(double *px,double *py,double *pq,long *pt)
{
    static long cl0=0,cr0=0;
    static double x=0.0,y=0.0,q=0.0;
    long cl,cr,ct;
    int dl,dr;
    double ds;

    ms_read_c(&cl,&cr,&ct);
    dl=cl-cl0;
    dr=cr-cr0;
    cl0=cl;
    cr0=cr;
    ds=DPC/2*(dr+dl);
    x += ds*cos(q);
    y += ds*sin(q);
    q += APC*(dr-dl);

    *px=x;
    *py=y;
    *pq=q;
    *pt=ct;
    if (k<DMAX)
    {
        xi[k]=10*x;
        yi[k]=10*y;
        qi[k]=1000*q;
        ti[k]=ct;
        k++;
    }    

}

void translate(double d,double vmax,double gain)
{
    double x0,y0,q0,x,y,q;
    long t0,t,nt;
    double cosq0,sinq0,dd,vc;
    int nc,nl,nr;

    estimate(&x0,&y0,&q0,&t0);
    cosq0=cos(q0);
    sinq0=sin(q0);
    x=x0;
    y=y0;
    q=q0;
    t=t0;

    do
    {
        dd=cosq0*(x-x0)+sinq0*(y-y0);
        vc=gain*(d-dd);
        if (ABS(vc)>vmax)
        {
            vc=(vc<0.0)?-vmax:vmax;
        }
        nc=vc*MMPS;
        ms_set_v(nc,nc);
        estimate(&x,&y,&q,&t);
        ms_read_v(&nl,&nr,&nt);
    } while (ABS(d-dd)>THRESH_D||ABS(nl)>THRESH_V||ABS(nr)>THRESH_V);
}

int translate2(double d,double vmax,double gain)
{
    double x0,y0,q0,x,y,q,g;
    long t0,t,nt;
    double cosq0,sinq0,dd,vc;
    int nc,nl,nr,s[3],check;

    int i=0;    /*センサ左中右切り替え用*/

    estimate(&x0,&y0,&q0,&t0);
    cosq0=cos(q0);
    sinq0=sin(q0);
    x=x0;
    y=y0;
    q=q0;
    t=t0;

    do
    {
        ms_ifr(s);

        /*過去WIDTH個のデータの平均でs[i]を判断*/
        for(i=0;i<3;i++)
        {
            // WIDTH 個のデータを捨てるために合計から引く
            sum_s[i] = sum_s[i] - data_s[i][bp] ;

            data_s[i][bp] = s[i] ;      // 最新データを保存

            sum_s[i] = sum_s[i] + s[i] ;         // 最新のデータで合計
            
            // 直近のデータを覚える場所を移動
            bp++ ;
            if ( bp >= WIDTH )
            {
                bp = 0 ;
            }
            
            // 直近wIdth個のデータ割る個数で出した平均が1以上の場合にsを1、1未満の場合sを0とする
            if((sum_s[i]/WIDTH)>=0.8)
            {
                s[i]=1;
            }else{
                s[i]=0;
            }
        }

        check = 1*s[0]+2*s[1]+4*s[2];
        ms_read_v(&nl,&nr,&nt);

        if (check)
        {
            for (g = 1; g > 0; g-=0.01)
                ms_set_v((int)(nl*g),(int)(nr*g));
            
            return check;   
        }
        
        dd=cosq0*(x-x0)+sinq0*(y-y0);

        vc=gain*(d-dd);

        if (ABS(vc)>vmax)
        {
            vc = (vc < 0.0) ? -vmax : vmax;
        }
        nc=vc*MMPS;
        ms_set_v(nc,nc);
        estimate(&x,&y,&q,&t);
        ms_read_v(&nl,&nr,&nt);
    } while (ABS(d-dd)>THRESH_D||ABS(nl)>THRESH_V||ABS(nr)>THRESH_V);
    return 0;
}

void rotate(double qd,double wmax,double gain)
{
  double x0,y0,q0,x,y,q;
  long t0,t,nt;
  double qq,wc,v,dw;
  int nl,nr;

  estimate(&x0,&y0,&q0,&t0);
  x=x0;
  y=y0;
  q=q0;
  t=t0;

  do{
  qq=q-q0;
  wc=gain*(qd-qq);
  if(ABS(wc)>wmax)
  {
    wc=(wc<0.0) ? -wmax:wmax;
  }
  dw=DISTANCE*wc;
  nl=-MMPS*dw;
  nr=MMPS*dw;
  ms_set_v(nl,nr);
  estimate(&x,&y,&q,&t);
  ms_read_v(&nl,&nr,&nt);
  }while (ABS(qd-qq)>THRESH_Q||ABS(nl)>THRESH_V||ABS(nr)>THRESH_V);
}

double seikika(double angle)
{
    while(angle>M_PI)   angle -= 2*M_PI;
    while(angle<=-M_PI)   angle += 2*M_PI;
    return angle;
}

void avoid(int j)
{
    if(j!=0)
    {
        translate(-REVERSE,VMAX,5);
        if(j&1) rotate(-60*D2R,WMAX,5);
        else rotate(60*D2R,WMAX,5);
        translate(REVERSE,VMAX,5);
    }
} 
\end{lstlisting}

\mysubsection{実行結果}
元々センサの入力が高感度であったため、障害物の検知が不安定であった。
しかし、センサの値が平均化することによってそのようなノイズが減り、
障害物検知が安定した。