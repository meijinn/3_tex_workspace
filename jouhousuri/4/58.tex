\mysection{58ページ目翻訳}

キューは、すべての挿入が「バック」と呼ばれる一方の端で行われ、すべての削除が「フロント」と呼ばれるもう一方の端で行われるリストです。 したがって、キューは「先入れ先出し」リストです。 図 2-7 はキューを示しています。
これらの定義を使用して、コンピューターにおけるグラフの表現について説明する準備が整いました。 表現の選択は、アルゴリズムの効率に影響を与えることがよくあります。

グラフの可能な表現の 1 つは、隣接行列を使用することです。 
$G$ を $V(G) = \{v_1, v_2, \dots, v_p\}$ の $(p, q)$ グラフとする。 
$G$ の隣接行列 $A = [a_{i,j}] $ は、次のように定義される $p\times p$ 行列です。

\begin{equation}
  a_{i,j}=
  \begin{cases}
    1  & if v_iv_j \in E(G) \\
    0  & otherwise.
  \end{cases}
\end{equation}

したがって、$A$ は、主対角線上のすべてのエントリが 0 である対称行列です。 グラフとその隣接行列を図 2-8 に示します。
$A$ は $p \times p$ 行列であるため、$p^2$ のメモリ位置をそのエントリに割り当てる必要があります。
$G$ が比較的辺の少ないグラフ (つまり、数値 q/p が小さい) の場合、その隣接行列の多くの位置に 0 が含まれます。
したがって、比較的少数の辺には異常に大量のメモリ スペースが必要になります。 
ただし、この問題は、グラフを隣接リストで表すことによって修正できます。

$V(G) = \{v_1, v_2, \dots, v_p \}$ のグラフ $G$ を考えます。 
$G$ の隣接リスト表現。便宜上、$v_i$ は単に $i ( 1 \leq i \leq 6)$。 $i$ 番目の隣接リストでは、
$i$ に隣接する頂点が番号順にリストされます。 実際、特定の頂点に隣接する頂点は常に数値順または
アルファベット順にリストされると仮定します。 行は NULL ポインターで終了します。

一般に、$(p, q)$ グラフ $G$ の隣接リストは、$(p+2q)\times 2$ テーブル $T$ で記述することもできます (図 2-8 を参照)。
$T$ の $(n, j)$ エントリは $T(n,j)$ で表されます。ここで、$1 \leq n \leq p+2q$ および $j = 1$ または 2 です。
$1 \leq n \leq p$ の場合、$T(n, 1)$ は空白になります。 $p+1 \leq n \leq p+2q$ の場合、$T(n,1)$ は隣接リストに表示される頂点を表します。