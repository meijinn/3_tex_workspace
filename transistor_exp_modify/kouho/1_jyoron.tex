\section{序論}
\subsection{研究の背景}
当時遠島を申し渡された罪人は、もちろん重い科を犯したものと認められた人ではあるが、決して盗みをするために、人を殺し火を放ったというような、獰悪な人物が多数を占めていたわけではない。高瀬舟に乗る罪人の過半は、いわゆる心得違いのために、思わぬ科を犯した人であった。有りふれた例をあげてみれば、当時相対死と言った情死をはかって、相手の女を殺して、自分だけ生き残った男というような類である。

そういう罪人を載せて、入相の鐘の鳴るころにこぎ出された高瀬舟は、黒ずんだ京都の町の家々を両岸に見つつ、東へ走って、加茂川を横ぎって下るのであった。この舟の中で、罪人とその親類の者とは夜どおし身の上を語り合う。いつもいつも悔やんでも返らぬ繰り言である。護送の役をする同心は、そばでそれを聞いて、罪人を出した親戚眷族の悲惨な境遇を細かに知ることができた。所詮町奉行の白州で、表向きの口供を聞いたり、役所の机の上で、口書を読んだりする役人の夢にもうかがうことのできぬ境遇である。

同心を勤める人にも、いろいろの性質があるから、この時ただうるさいと思って、耳をおおいたく思う冷淡な同心があるかと思えば、またしみじみと人の哀れを身に引き受けて、役がらゆえ気色には見せぬながら、無言のうちにひそかに胸を痛める同心もあった。場合によって非常に悲惨な境遇に陥った罪人とその親類とを、特に心弱い、涙もろい同心が宰領してゆくことになると、その同心は不覚の涙を禁じ得ぬのであった。

そこで高瀬舟の護送は、町奉行所の同心仲間で不快な職務としてきらわれていた。

\subsection{研究の目的}
いつのころであったか。たぶん江戸で白河楽翁侯が政柄を執っていた寛政のころででもあっただろう。智恩院の桜が入相の鐘に散る春の夕べに、これまで類のない、珍しい罪人が高瀬舟に載せられた。

それは名を喜助と言って、三十歳ばかりになる、住所不定の男である。もとより牢屋敷に呼び出されるような親類はないので、舟にもただ一人で乗った。

護送を命ぜられて、いっしょに舟に乗り込んだ同心羽田庄兵衛は、ただ喜助が弟殺しの罪人だということだけを聞いていた。さて牢屋敷から棧橋まで連れて来る間、この痩肉の、色の青白い喜助の様子を見るに、いかにも神妙に、いかにもおとなしく、自分をば公儀の役人として敬って、何事につけても逆らわぬようにしている。しかもそれが、罪人の間に往々見受けるような、温順を装って権勢に媚びる態度ではない。

庄兵衛は不思議に思った。そして舟に乗ってからも、単に役目の表で見張っているばかりでなく、絶えず喜助の挙動に、細かい注意をしていた。

その日は暮れ方から風がやんで、空一面をおおった薄い雲が、月の輪郭をかすませ、ようよう近寄って来る夏の温かさが、両岸の土からも、川床の土からも、もやになって立ちのぼるかと思われる夜であった。下京の町を離れて、加茂川を横ぎったころからは、あたりがひっそりとして、ただ舳にさかれる水のささやきを聞くのみである。

夜舟で寝ることは、罪人にも許されているのに、喜助は横になろうともせず、雲の濃淡に従って、光の増したり減じたりする月を仰いで、黙っている。その額は晴れやかで目にはかすかなかがやきがある。

庄兵衛はまともには見ていぬが、始終喜助の顔から目を離さずにいる。そして不思議だ、不思議だと、心の内で繰り返している。それは喜助の顔が縦から見ても、横から見ても、いかにも楽しそうで、もし役人に対する気がねがなかったなら、口笛を吹きはじめるとか、鼻歌を歌い出すとかしそうに思われたからである。

庄兵衛は心の内に思った。これまでこの高瀬舟の宰領をしたことは幾たびだか知れない。しかし載せてゆく罪人は、いつもほとんど同じように、目も当てられぬ気の毒な様子をしていた。それにこの男はどうしたのだろう。遊山船にでも乗ったような顔をしている。罪は弟を殺したのだそうだが、よしやその弟が悪いやつで、それをどんなゆきがかりになって殺したにせよ、人の情としていい心持ちはせぬはずである。この色の青いやせ男が、その人の情というものが全く欠けているほどの、世にもまれな悪人であろうか。どうもそうは思われない。ひょっと気でも狂っているのではあるまいか。いやいや。それにしては何一つつじつまの合わぬことばや挙動がない。この男はどうしたのだろう。庄兵衛がためには喜助の態度が考えれば考えるほどわからなくなるのである。

\subsection{本論文の構成}
本論文の構成を説明する.本論文は森鴎外の高瀬舟を使用している.後半では宮沢賢治のポラーノの広場を使用する.