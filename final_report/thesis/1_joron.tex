\chapter{序論}
\section{研究背景}

近年、様々な自律移動ロボットの研究開発が行われ、徐々に人間の生活環境に進出し始めている。
従来は、工場における産業用無人搬送車が主体であったが、活躍の場がオフィス、家庭、さらには屋外へと広がりつつある。
今後、ロボットを用いた省力化や危険な仕事の代替を進めるためには、自律移動ロボットが活動できる領域が人間の生活環境へと拡大していくことが期待される。

一例として、自律移動の応用技術として交通課題の解決策として自動運転システムの実用化が進んでいる。現在は法定上遠隔地からの監視・運用を必要とする遠隔型自動運転システムが想定されている。
遠隔型自動運転では、FPV(First Person View)における車速の制御に関して、ドライバの体感速度変化を促すための視覚効果に関する研究が行われている。
視覚効果は、定式化に関して未知数であり、体感速度変化の評価はドライバ自身の知覚の個人差を含めた主観で考えられるため、定性的解析に委ねられている。

自律移動ロボットの実現に関して、工場のように限定された環境では、環境をロボットのために整備することが可能であった。
しかしロボットの活動領域を広げるためには、環境をロボットに合わせて整備するのではなく、ロボットが様々な環境に適応することが求められる。
人間の生活環境は整備された工場と比較して複雑であり、不確実性が高くなる環境である。この不確実性への対処が、重要な課題となる。
特に、ロボットに必要となる知覚、計画、制御のうち、不確実性に最も深く関わるのが知覚(計測と認識)である。
移動ロボットを対象として考えると、重要な知覚の機能として自己位置推定と地図生成(SLAM: Simultaneous Localization and Mapping)がある。
自己位置推定は、ロボットが自身の位置(方位を含む)を認識しながら目的地まで移動するために必要となる。
また地図は、自己位置推定や経路計画に必要である。人手による地図の生成は多大な工数が必要なため、ロボットが自動的に生成することが望ましい。
移動ロボットの自己位置推定と地図生成において不確実性に対処するには、確率論に基づく手法が有効であることが知られている。
確率論に基づくことで、センサデータやコンピュータが持つモデルの不確実性を明示的に表現すると共に、事前確率の情報を用いてより確実性の高い推定を行うことが可能となる。
一般に、SLAMでは、外界センサであるLiDARに加えて、ロータリーエンコーダ、慣性測定ユニット(IMU)等の内界センサによるロボットの移動量の内部情報(オドメトリ)を用いることで、マッピングを行うことが主流とされているが、オドメトリを必要としない、オドメトリフリーのSLAM手法も提案されている。

自律移動ロボットの開発プロセスにおいて、実環境における実証実験が数多く行われている。
特に、屋外における自律移動においては、シミュレーションと実環境では、制約や環境条件が全く異なるため、実環境のみでしか得ることのできない知見が多く、自律移動の実用化に向けては、実機を用いた実証実験が必要とされている。
実証実験に代表されるつくばチャレンジでは、自律移動ロボットの設計において、要素技術であるLiDARを中心とした、センサ系の使用が主となり、複数のセンサとの融合(センサフュージョン)による環境認識の高精度化が図られている。
それと同時に、自律移動ロボットの実用化に向けて、使用されるセンサの選定、用途に応じて最適化する試みが行われている。
特に、ロボットに搭載するセンサの使用数を少なくすることは、ロボットに課される制約・条件、ソフトウェア開発の工数減、保守性の向上のために、考慮すべき点と述べられている。
自律移動チャレンジの参加車両の形態は多様であり、差動二輪型が一般的であるが、パーソナルビークル等の使用としては、4輪車両型の自律移動等も見られる。
自律移動ロボットは、歩行者等と同じ空間に共存するロボットとしての位置づけであり、比較的低速であることが特徴であり、LiDARによる環境認識の比重が高い。
対して、自動運転では、リアルタイム性が重視され、カメラや深度センサ等のイメージセンサ等のコンピュータビジョンとLiDARセンサによるSLAMがそれぞれ同じ割合での併用が標準となり、センサ融合に関する技術について言及されている。

\section{本研究の目的と特徴}
本研究では、ROS\verb|(Robot Operating System)|
を用いた4輪車両型自律移動ロボットシステムの開発を目的として、一つ目に、LiDARのみによるSLAMによって生成された環境地図の評価と、二つ目にステアリングコントローラによる、ハンドルフットペダル型インターフェースによる遠隔操作を用いた、手動運転自動運転の切り替えシステムの開発を行う。
ROSに実装されたSLAMメタパッケージである、
slam\_gmapping
による環境地図構築とNavigationメタパッケージである、Navigation Stack を用いて、Waypoint Navigationにより、中継目的地(waypoint)を経由して、
最終目的地まで自律移動を行い、最終目的地に到着した時点で、手動運転に切り替えることで、運転操作の引継ぎを実現するシステムを構築する。
三つ目に、遠隔型自動運転システムのFPV操作に着目した、自動車の映像における、運転視野角や、映像画角の変化による体感速度のモデルの定量化を目的として、体感速度モデルの定式化を提案する。
自律移動ロボットの要素技術として用いられているLiDARのメーカーとして有名な北陽電機株式会社は、中之島チャレンジに参加し、自律移動ロボットの実機を開発して、実際にフィールドで議論を行うことで、新製品の開発を行っている。
自律移動ロボットの要素技術であるLiDARの開発においても、実証実験が不可欠である。
本研究の位置付けは、遠隔型自動運転における車速認知の定式化に関する仮説の立証と、自動運転(特に4輪型車両の自律移動)において、センサ数の少ないLiDARに重きを置いた制御理論の確立とする。

\section{本論文の構成}

本論文は全5章から構成される。以下に各章の概要を述べる。

第1章では、本研究の背景と目的及び各章の構成を述べた。

第2章「FPV車両操作の体感速度変化率の定式化」では、ドライバの体感速度変化を促すための視覚効果に関する仮説を提案する。具体的には、体感速度の変化率を支配するパラメータに着目し、走行映像の視聴環境ドライバの視野角と、走行映像のクロップ率をパラメータとしたときの単位時間当たりの映像ピクセルの移動量を体感速度の変化率として、クロップ率・視野角の減少と増加の特性を一つのモデルとして定式化した。

第3章「LiDARのみのSLAMで作成された環境地図の評価」では、内界センサによるオドメトリを用いず、ROSで実装されている各LiDAR SLAMパッケージを用いたLiDARのみによる環境地図作成システムを開発し、オドメトリフリーの手法に対して、オドメトリが必要な手法にLiDARから得るレーザオドメトリを用いて比較することで、LiDARのみによるSLAMの性能を明らかにした。

第4章「自動運転システムの開発」では、測距センサ(LiDAR)による測距データのみを用いたSLAMによって作成された環境地図を用いて、4輪車両型RCカーのLiDARのみによる自己位置推定とWaypoint Navigationの有用性の検証を行った。特に、4輪車両特有の動作である、切り返しの実現性、自動運転システムとして機能をする場合の遠隔操作のユーザビリティと、手動・自動運転のタスクの切り替え機能の動作に着目をした評価を行った。

第5章「結論」では、本研究で得られた結果とその成果について述べる。

