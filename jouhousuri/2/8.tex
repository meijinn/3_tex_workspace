\mysection{8ページ目翻訳}

しかし、第6章で検証するように、$r$ と $p$がともに奇数ではなく、$0 \leq  r \leq  p-1$である場合、必ず次のような位数$p$の$r$-正則グラフが存在する。
グラフ G の補グラフGh とは、グラフ$G$が$V(G) = V(\overline{G})$ で、$uv$ が $G$ の辺でない場合にのみ、$uv$ が $\overline{G}$ の辺となるグラフのことである。
図1-8にグラフとその補数を示します。次数$p$のグラフ$G$において$v$が次数$n$の頂点である場合、$G$における$v$の次数は$p-n-1$であることがわかる。
したがって、$\overline{G}$は$G$が正則である場合にのみ正則である。
\\

問題点 SET 2

\begin{enumerate}
  \item 次数$n \geq 2$のグラフはすべて、同じ次数の頂点を少なくとも2つ持つことを証明せよ。
  \item 次数5のグラフ$G$で、$G$の2つの頂点がそれぞれ近傍にある、という性質を持つグラフの例を挙げよ。
        そのグラフを描き、その大きさと、各頂点の次数を明記する。
  \item 以下に示すグラフGの頂点$v_1, v_2, \ldots, v_6$の次数を決定し、$\sum_{i=1}^{6} \deg v_i$を計算する。これを用いて$G$の大きさを決定せよ。
  \item 頂点が1, 2, 2, 3, 4度である次数5のグラフを作る。このグラフの大きさはいくらか?
  \item ハービーは自分の家で開かれるパーティーに5人の友人を招待した。
  彼らが全員到着した後、ハービーはパーティーに参加している人のうち何人を知っているかを尋ねた。
  5人はそれぞれ違う答えをした。これは可能か?
\end{enumerate}