\mysection{48ページ目翻訳}

bubblesortアルゴリズムを使用する場合、n個の単語のリストを最初に通過させるには、n -1個の比較が必要です。
アルファベット順の最初の単語がリストの一番上に置かれると、次のパスでは一番下のn - l個の単語を見て、
n 2個の比較を行う。このように続けると、比較の総数は最大でn(n-1)個になることがわかる。
2 + 3 + 2 + 1 =なので、複雑さはO(n)となる。これは、逐次ソートと同じ複雑さである
(n- 1) + (n-2) + となります。
のアルゴリズムと同じ複雑さです。別のソートアルゴリズムは分割統治法を用いており、
ムネージソートとして知られている。n> 2の場合、このアルゴリズムはまずn個の単語からなる
リストL: W(), W(2), ..., W(n)を、Lの約半分の長さの2つのサブリストLiとL2に分割します。 
より具体的にはL|:W(1), W(2),..., W(3I)とLz:W(l+1)、W(L3|+2)、...、W(n)である。次に、アルゴリズムは、自分自身を呼び出すことによって、LiとLをアルファベット順にソートする。
最後に、2つの(アルファベット順の)リストが「マージ」される。これらのリストのそれぞれから最初の要素が比較される。
これらの2つの単語の間で最初にアルファベット順に並んだ単語は、アルファベット順にLの最初の単語となる。
この手順を残りの2つのサブリストで繰り返し、L上のアルファベット順の次の単語を決定する。
なお、2つのサブリストLiとL2をマージするためには、最大でn l回の比較が必要である。
問題5では、マージソートに必要な比較回数は最大O(n log n)であることを示せということだ.
他にも多くのソートアルゴリズムがある。その中で興味深いのは、クイックソート([1, 4]を参照)とヒープソート([1]を参照)である。クイックソートの複雑さはO (n)であり、シーケンシャルソートやバブルソートと同じであるため、「クイックソート」という名称は少し語弊があるような気がします。一方、ヒープソートは複雑さがO(n log n)である。もちろん、wrono Mseのこれらのソートアルゴリズムは定義上すべて効率的である。
第1.5節では、p個の非負整数の並びがグラフであるかどうかを判定する手順を説明した。ここで、これをアルゴリズムとして正式に再定義する。
アルゴリズム2.6.
[p (2 1)個の非負整数の並びが図形的であるかどうかを判定する] 。
1.もし数列にp -1を超える項があれば、その数列は図形的でない。
図式的でない。
2.数列のすべての整数が0である場合、数列は図形的である。もし、数列に負の整数が含まれていれば、その数列は図形的でない。そうでない場合は、ステップ 3 に進みます。
必要であれば、現在の数列の数字を並べ替え、nou 4.最初の数字(n)を数列から削除し、次の数列から1を減算する。
n個の数字から1を引く、ステップ2に戻る。
ステップ1、2の複雑さはO(p)である。ステップ3では、P個の数字の並べ替えが必要な場合がある。例えば、バブルソートアルゴリズムを使用する場合、ステップ3では0(p)個の比較が必要になる。ステップ4が発生するたびに、最大でp - 1の減算が発生する。