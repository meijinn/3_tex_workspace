\mysection{55ページ目翻訳}

アルゴリズム 2.7。

[スーツケースに詰めるアイテムを $n$ の中から選択する貪欲なアルゴリズム。 項目は $1、2、\dots、n$ で表され、項目 $i$ は重み $w(i)$ と値 $v(i)$ を持ちます。 さらに、r は重量制限を示し、S は梱包されるアイテムのセットを示します。]

1. $v(1) \leq v(2) \leq \cdots \leq v(n)$ になるように項目を並べ替えます。

2. [梱包するアイテムの集合 S を Φ として初期化する。 変数合計重量と合計値は 0 に初期化されます。]

2.1S←Φ。

2.2 総重量 ← 0.

2.3 合計値 ← 0.

3. i = 1 ~ n の場合

[新しい総重量が重量制限を超えない場合、梱包する品目のセット S に品目 i を含めます。 S に i を追加すると、合計重量と合計値が増加します。]
総重量 $+ w(i) \leq r$ の場合、

(a) $S ← S \cup \{i\}$。

(b) 総重量$ ← +w(i)$

(c) 合計値 $← 合計値 + v(i)$。

4. 出力 S、合計値。

ステップ 1 では、値が増加しない順に $n$ 項目を並べ替える複雑さは、もちろん並べ替えアルゴリズムの選択によって異なります。 たとえば、バブルソート アルゴリズムを使用する場合、ステップ 1 の複雑さは $O(n^2)$ になります。 ステップ 3 と 4 の複雑さは $O(n)$ であるため、アルゴリズム 2.7 の複雑さは $O(n^2)$ になります。 したがって、アルゴリズム 2.7 は効率的なアルゴリズムです。
具体的な例として、8つの商品を梱包したいとします。 それらの重みと値を図 2-5 に示します。 さらに、スーツケースの制限重量が 44 ポンドであるとします。つまり、$r=44$ となります。 次に、アルゴリズム 2.7 に従って梱包される品目を図 2-5 に示します。
したがって、アルゴリズムはアイテム 1、2、5、および 8 を選択しました。これらのアイテムの総重量は 44 ポンド、合計値は 25 です。ただし、この選択は最適な解決策ではありません。 たとえば、アイテム 1、3、4、および 6 を選択した場合、合計の重みは 44 になり、合計値は 28 になります。これは、貪欲なアルゴリズムは効率的ですが、可能な限り最高の値を与える必要はないことを示しています。 問題の解決策。 最適なソリューションを段階的に導き出す唯一の方法が非効率なアルゴリズムを使用することである場合さえあります。