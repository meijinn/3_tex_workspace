\mysection{45ページ目翻訳}

を意味し、xの天井(x])はx以上の最小の整数である。xがそれ自体整数である場合、x] =|x=Xである。例えば、~ さらに、~である。
逐次探索アルゴリズムは、リスト上の最初の単語から開始する。これに対して、バイナリサーチアルゴリズムは、リストの途中から開始する。
バイナリサーチアルゴリズムは、リストの「mniddle」単語を比較して、それがKEYかどうかを判断する。 この単語がKEYでなく、KEYがこの真ん中の単語にアルファベット的に先行する場合、
KEYはリストの最初の単語と中間単語の中間にある単語と比較され、それ以外は、KEYはリストの中間単語と最後の単語の中間にある単語と比較される。
リストが$n$個の単語で構成されている場合、中間単語とは、|の位置に現れる単語を意味する。仮に$n=25$とする。
すると、$W(13)$はリストの真ん中の単語であり、$W(13)$とKEYを比較することになる。もちろん、$W(13)=KEY$であれば、KEYはリスト上にあり、
13番目の単語である。$W(13)$ KEYの場合、KEYが$W(13)$よりもアルファベット的に先行しているかどうかをチェックします。これをKEY$< W(13)$と書きます。
もし、実際にKEY $> W(13)$であれば、次にKEYと$W(19)$を比較します。
例えば、DOORという単語が次のリストにあるかどうかを判断したいとします。
という単語があるかどうかを調べたいとする:

W(1)
W(2)
W(3)
W(4)
W(5)
W(6)
W(7)
W(8)
W(9)
W(10)
= アロー
= ボール
= 車
= ドア
= 足
= ハンド
= ラダー
= ネット
= パン
= TENT

二値探索アルゴリズムを用いて、まず$W(5)=$FOOTがキーワードのDOORであるかどうかを確認する。そうではありません。D0OR$<$FO0Tなので、次にサブリスト$W(1), W(2), W(3), W(4)$の中間語である
$W(2) = $BALLを考える。BALL D0ORだがDOOR$>$BALLなので、次にDOORとサブリスト$W(3), W(4)$の中間の単語、
すなわち$W(3)=$CARを比較してみる。CAR$*$DOOR、しかしDOOR $>$ CARなので、残る単語はただ一つ、すなわち$W(4)$となり、
単語を発見したことになる。ここで、バイナリサーチアルゴリズムを正式に提示する。
アルゴリズム2.3(バイナリサーチアルゴリズム)。WOD、We、...、Wnの$n$個の単語を検索し、もし検索できたなら、
リスト上の位置を表示する)。[10 与えられた単語KEYがアルファベット順のリストに現れるかどうかを判定する。
1. fとlは、検討中のサブリスト上の単語の最初と最後の位置を表す。初期状態では、$f=l$、$l=n$である]。1.1 $f-1$.
1.2 l n.