\mysection{81ページ目翻訳}

定理 3.10
$T$を接続グラフ$G$の深さ優先探索木とし、$u$を$T$の根ではない頂点とする。 
$u$ は、$l(v) \leq dfi(u)$ のような子 $v$ が $T$ にある場合に限り、$G$ のカット頂点になります。

証拠。
まず、$u$ が $T$ のルートではなく、$u$ が $G$ のカット頂点であると仮定します。
 $r$ ($u$) を $T$ のルートとします。 この場合、$r$ は $u$ で $G$ のブランチ $B_1$ に属します。 
 $u$ はカット頂点であるため、未訪問の頂点がまだある間に、深さ優先検索は最終的に $u$ に到達する必要があります。 
 $v$ を、$B_1$ に属さない $u$ に隣接する最初の頂点とする。 $v$ がブランチ $B_2$ に属しているとします。 
 $u$ と異なる $B_2$ の頂点は $G - V(B_2)$ の頂点に結合されていないため、 $l(v) \leq dfi(u)$ となり、$uv$ は $l(v) \leq dfi(u)$ を持つ
 $T$ の端。 $r$ を $T$ のルートとし、$S_2$ を $v$ をルートとする $T$ の最大部分木の頂点の集合を表すものとします。 
 定理 3.8 より、$S_2$ の頂点と $G-(S_1\cup S_2 \cup \{u\}$ の頂点を結ぶ辺は存在しません。

$G$ にエッジ $x_1x_2$ が含まれているとします ($x_i \in S_i (i = 1, 2)$)。 その場合、$x_2x_1$ は
バックエッジ、$dfi(x_1) dfi(u)$ でなければなりません。 $T$ の $v-x_2$ パスに続いてエッジ $x_2x_1$ を取ると、
$l(v) \leq dfi(x_1)$ であることがわかります。 したがって、$l(v) < dfi(u)$ となり、矛盾します。
したがって、$S_1$ の頂点と $S_2$ の頂点を接続する $G$ 内のすべてのパスは $u$ を通過します。
これは、$u$ が $G$ のカット頂点であることを意味します。

グラフにブリッジがあるかどうかを知ることが重要な場合があります。 
第 11 章では、そのような状況に遭遇します。 深さ優先探索を利用して、橋の特徴を表示することもできます。
エッジ $e$ がグラフ $G$ のブリッジである場合、$e$ は、$e$ が $G$ のすべての深さ優先探索フォレストに属することに注目してください。
カット頂点の場合と同様に、接続されたグラフのブリッジを特徴付けるだけで済みます。