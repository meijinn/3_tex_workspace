\mysection{16ページ目翻訳}

次数列の最小項$deg V_p$を$G$の最小次数と呼び、$\delta(G)$と表記し、
最大項$\deg v_1$を最大次数と呼び、$\Delta(G)$と表記している。
例えば、図のグラフ$G$は次数列4, 4, 3, 2, 2, 1, 0を持つ。したがって、このグラフでは$\delta(G)=0$で $\Delta(G) = 4$となる。

$d_1, d_2, \ldots, d_p$をあるグラフの次数列とすると、
必然的に $\sum_{i=1}^{p} d_i$ は偶数で、$0 \leq  d_i \leq  p- 1$したがって $1 \leq i \leq p$.
しかしながら、$\sum_{i=1}^{p} d_i$ が偶数で、
$0 \leq d_i \leq p-1$ for $1 \leq i \leq p$となるような
整数の列$s: d_1, d_2,\ldots, d_p$が与えられた場合、
sがあるグラフの次数列であるという保証はない。

例えば、$s:5, 5, 3, 2, 1, 0$ は、和が偶数で各項が最大5である6つの非負整数の列であるが、$s$はいかなるグラフの次数列でもない。
しかし、次数5の頂点は、次数0の頂点を含む$G$のすべての頂点に隣接しており、これはありえないことである。
しかし、Havel [6]とHakimi [4]による、どの非負整数の列がグラフの次数列であるかを決定することができる、便利な結果がある。
非負整数列があるグラフの次数列である場合、それをグラフ列と呼ぶ。


例えば、$s: 5,5,3,2,1,0$ は和が偶数で各項が最大5である6つの非負整数の列であるが、
$s$はいかなるグラフの次数列でもない。しかし、度数5の頂点は、度数0の頂点を含む$G$のすべての頂点に隣接しており、これは不可能である。
しかし、Havel [6]とHakimi [4]による、あるグラフの次数列であれば、非負整数のどの列がグラフ的であるかを決定することができる有用な結果が存在する。

\begin{description}
  \item[定理1.2] (Havel-Hakimi)
  非負整数の数列$s:d_1, d2, \ldots ,d_p$で、$d_1, d_2, \ldots, d_p,$である $p \geq 2, d_1 \geq 1$は、以下の場合にのみ、グラフ的である$s$はグラフ的である
\end{description}

この結果の証明を紹介する前に、この定理が何を言っているのかが分かっていることを確認しておこう。
まず、数列$s_1$は、$s$から最初の項$d_1$を削除し、
$s$のちょうど次の$d_1$項から1を引くことによって得られる。
この定理は、$s$がグラフかどうかを判断するには、
$s_1$がグラフかどうかを判断すればよいことを教えている。
しかし、$s_1$は$s$より項数が少なく、$s_1$の項の中には少し小さいものもあるので、
この作業はもっと簡単なはずである。