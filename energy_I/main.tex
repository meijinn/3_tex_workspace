\documentclass[10pt,a4j,dvipdfmx]{jarticle}
%---------------------------------------------------
\usepackage{hyperref}
\usepackage{pxjahyper}
\usepackage{bm}
\usepackage{graphicx}
\usepackage{amssymb,amsmath}
\usepackage{ascmac}
\usepackage{float}
\usepackage{setspace}
\usepackage[dvips,usenames]{color}
\usepackage{colortbl}
\usepackage{algorithm}
\usepackage{algorithmic}
\usepackage{setspace}
\usepackage{subfigure}
\usepackage{here}
\usepackage[deluxe,bold]{otf}
\usepackage[haranoaji]{pxchfon}
\usepackage{redeffont}
\usepackage{listings,jvlisting} %日本語のコメントアウトをする場合jvlisting(もしくはjlisting)が必要
\usepackage{booktabs}
%---------------------------------------------------
% \definecolor{bl}{rgb}{0.94,0.97,1}
% \definecolor{gr}{rgb}{0.5,0.5,0.5}
% \makeatletter
% \def\section{\newpage\@startsection {section}{1}{\z@}{2.3ex plus -1ex minus -.2ex}{2.3 ex plus .2ex}{\Large\bfseries}}
% \makeatother
%---------------------------------------------------
\setlength{\textwidth}{160truemm}
\setlength{\textheight}{240truemm}
\setlength{\topmargin}{-14.5truemm}
\setlength{\oddsidemargin}{-0.5truemm}
\setlength{\headheight}{0truemm}
\setlength{\parindent}{1zw}
\setlength{\abovedisplayskip}{-2pt} % 数式上部のマージン
\setlength{\belowdisplayskip}{-2pt} % 数式下部のマージン
%---------------------------------------------------
\setstretch{1.2}
%---------------------------------------------------
\renewcommand{\subfigtopskip}{5pt}	% 図の上の隙間。上図の副題と下図の間。
\renewcommand{\subfigbottomskip}{0pt} % 図の下の隙間。副題と本題の間。
\renewcommand{\subfigcapskip}{-6pt}	% 図と副題の間
\renewcommand{\subcapsize}{\scriptsize} % 副題の文字の大きさ
\newcommand{\mysection}[1]{\newpage\vspace{-20pt}\section{#1}}
\newcommand{\mysubsection}[1]{\vspace{-20pt}\subsection{#1}}
\newcommand{\mysubsubsection}[1]{\vspace{-10pt}\subsubsection{#1}}
\renewcommand{\lstlistingname}{ソースコード}
%---------------------------------------------------
% ヘッダーとフッターの設定
\usepackage{fancyhdr}
\rhead{\leftmark}
\chead{}
\lhead{\rightmark}
\cfoot{\thepage}

\rfoot{}
\begin{document}
%---------------------------------------------------
\setlength{\abovedisplayskip}{1.5pt} 
\setlength{\belowdisplayskip}{0pt}
%---------------------------------------------------
%ここからソースコードの表示に関する設定
\lstset{
  basicstyle={\ttfamily},
  identifierstyle={\small},
  commentstyle={\smallitshape},
  keywordstyle={\small\bfseries},
  ndkeywordstyle={\small},
  stringstyle={\small\ttfamily},
  frame={tb},
  breaklines=true,
  columns=[l]{fullflexible},
  numbers=left,
  xrightmargin=0zw,
  xleftmargin=3zw,
  numberstyle={\scriptsize},
  stepnumber=1,
  numbersep=1zw,
  lineskip=-0.5ex
}
%ここまでソースコードの表示に関する設定
%---------------------------------------------------

\thispagestyle{empty}
\vspace*{2cm}
\thispagestyle{empty}
\begin{spacing}{1}
\begin{center}
{\Large 明石工業高等専門学校専攻科 \\[1truecm]
エネルギー工学I} \\[3.5truecm]
\Huge 課題1(再提出分) \\
[5truecm]
\Large ME2208 \CID{8705}橋 尚太郎 \\
(機械・電子システム工学専攻1年) \\[1truecm]
提出年月日: \today
\end{center}

\newpage
\pagenumbering{roman}
\end{spacing}
\clearpage
\pagenumbering{arabic}
\pagestyle{fancy}
\setlength{\headheight}{5truemm}

\mysection{仕様と実現方法}
\begin{enumerate}
  \item たまにある単純作業の繰り返しの作業は何か \\
  光学式測距センサ(LiDAR)で得た距離データ(センサデータ)のばらつきを統計的に処理したい。
  ロボットの自律移動を行うために、ロボットの自己位置・速度の推定を行う必要がある。
  一例として、LiDARを用いてランドマークとの距離を計測し、自己位置を得る手法が提案されている。
  ロボットに搭載するセンサデータは、センサ内部の状態や周囲の環境等、さまざまな影響を受けて出力される。
  \item どのようなデータを可視化したいか\\
  センサデータに含まれる誤差(ノイズ)の傾向を可視化したい。

  \item どのような機能があればそれが実装できそうか\\
  統計処理で扱う関数やモジュールと、それを可視化するためのツールがあれば実装できる。
  センサデータを解析するために、pythonにおいて、以下に示したツールを用いることが想定される。
    \begin{itemize}
      \setlength{\parskip}{0cm} % 段落間
      \setlength{\itemsep}{0cm} % 項目間
      \item 距離データが格納されたファイルの読み込み
      \item 配列等への格納
      \item 確率分布、平均、分散、標準偏差、正規分布を扱うためのモジュール
      \item 数値の可視化 (グラフ化ツール)
    \end{itemize}
  順にこれらを実装することで、測距データに含まれるノイズの傾向が明らかになる。
\end{enumerate}

\newpage
\pagestyle{plain}
\bibliographystyle{jplain}
\addcontentsline{toc}{section}{参考文献}
\begin{thebibliography}{3}
\setlength{\parskip}{0cm} %enumerateのマージン
\setlength{\itemsep}{0cm}

\bibitem{kougi_shiryo}
令和4年度
エネルギー工学I
講義資料

\end{thebibliography}

\end{document}