\mysection{47ページ目翻訳}


2.3 ソートアルゴリズム

セクション2.2で説明した検索アルゴリズムでは、与えられたリストがアルファベット順であると仮定した。この節では、与えられた単語のリストをアルファベット順に並べるための2つのアルゴリズムについて説明する。最初のものは、順次ソートまたは選択ソートアルゴリズムと呼ばれ、アルゴリズム2.1を用いて、リスト上のアルファベット順の最初の単語を探し出すものである。
アルゴリズム2.4(逐次ソートアルゴリズムn)。[n個の単語からなるリストWl)、W(2)、...、W(n)をアルファベット順に並べる]。
1.k = 1 から n -1 まで
(a) アルゴリズム2.1を適用して、サブリストW(k), W(k+ 1),..., W(n) から、最初の単語であるFIRST = W(p) を、アルファベット順に求める。(b)[サブリストのアルファベット順の最初の単語と、サブリストの最初の単語を入れ替える]。
W(p) - W(k).
W(k) < FIRST.2.[単語はアルファベット順に出力される] 。
i=1〜nの場合
W(i)を出力する。
順次ソートのアルゴリズムをリストで説明する:w(1) = fence, w(2) = side, w(3) = car, w(4) = gate.n = 4なので、k = 1、k = 2、k = 3に対応するように、ステップlaと1bを3回実行する。


オリジナルリスト(= )
フェンス
サイド
カーゲート
ステップ1(k
CAR
サイドフェンス
ゲート
= )
ステップ1 (k = 2)
CAR
フェンス
サイドゲート
ステップ1 (k = 3)
CAR
フェンス
ゲート 側面
アルゴリズム2.4の複雑さを理解するために、まず、アルゴリズム21は最大でn - 1個の比較を必要とすることを思い出してください。アルゴリズム2.1を2回目に適用した場合、サブリストはn - l個の単語を持つ。このようにすると、必要な比較は最大でも (n - 1) + (n -2)+ **+3+2+1 = () = n(n 1)/2 比較となる。(この式は数学的帰納法で検証できる。) したがって、アルゴリズム2.4の計算量は
$O(n)$である。単語のリストをアルファベット順に並べるもう一つの一般的なアルゴリズムは、リストの最後から始めて、リストを通過するたびに連続する2つの単語を比較します。2つの単語がアルファベット順でない場合、単語は交換される。同じリストを用いて、この方法を説明します。

元のリスト
フェンス
サイド
車のゲート
一次通過
フェンス
カーサイド
ゲート
セカンドパス
CAR
フェンス
ゲートサイド
カー
フェンス
ゲートサイド
サードパス
CAR
フェンスゲート
サイド
このアルゴリズムは、アルファベット順の最初の単語がリストの先頭に「バブルアップ」することから、バブルソートアルゴリズムと呼ばれています。また、時折、交換ソートアルゴリズムとも呼ばれることがある。
アルゴリズム2.5(bubblesortアルゴリズム)。[n個の単語のリストW)、W(2)、...W(n)をアルファベット順に並べる]。
1.

j = l から n 1 まで
[最初のj -1個の単語をアルファベット順に並べると、WU), WO + 1), ..., W(n)の単語だけがソートされる必要がある。]i=1〜n-jの場合
[リストWy)、Wy + 1)、......の連続した単語をチェックする。W(n)を下から順にチェックし、所望の交換を行う]。W(n +1 - i)<W(n -i)であれば、(a) W(n - i)をTEMPする。
(b) W(n - i) W(n +1 - i).
() W(n + 1 -)=TEMP.2. (単語はアルファベット順に出力される。)k=1〜nの場合
W(k)を出力する。
フェンス
サイド
車
ゲート
車
フェンス
サイド
ゲート