\mysection{22ページ目翻訳}

1.7 切断頂点と橋
連結グラフの頂点、辺、部分グラフには、
特筆に値する次のような種類があります。
これらについては、本節で説明する。

グラフ$G$の頂点$v$は、$k(G-v) > k(G)$のとき、
切断頂点と呼ばれる。
つまり、$G-v$が切断されている場合、
$v$は連結グラフ$G$の切断点となる。
図1-19のグラフGの頂点$v_3$と$v_s$は、
このグラフの唯一の切断点である。
ここで、切断頂点の類似物である辺について考えてみる。
グラフ$G$の辺$e$は、$k(G-e) > k(G)$のとき、
橋と呼ばれます。
したがって、連結グラフ$G$の辺$e$は、$G-e$が切断されていれば
ブリッジである。図1-19のグラフ$G$の辺$e=v_5v_6$は、
このグラフの唯一のブリッジである。
$v$が連結グラフ$G$の切断頂点である場合、
$G-v$には2つ、あるいはより多くの構成要素を含む。
しかし、$e$が$G$のブリッジである場合、
$G-e$はちょうど2つの構成要素を含む。
次の定理は、グラフの辺のうち、橋であるものを特徴づけるものである。
\begin{description}
  \item[定理1.4.] 連結グラフ$G$の辺$e$は、$e$が$G$のサイクル上にない場合に限り、$G$のブリッジとなる。
\end{description}