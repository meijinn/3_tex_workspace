\mysection{85ページ目翻訳}

ステップ6では、最大でもその頂点に付随するすべての辺について頂点の最低点の値が更新されるため、
ステップ6が実行される合計回数は $\sum _{v\in V(G)} \deg v = 2q$ に比例します。
したがって、ステップ 6 の計算量は $O(q)$ になります。 ステップ 7 の計算量は $O(p)$ です。
そして、ステップ 8 ~ 11 は各頂点から最大 1 回バックトラックするため、これらのステップの計算量は $O(p)$ になります。 
ステップ 12 では各頂点を最大 1 回スキャンします。これらのステップの計算量は $O(p)$ です。
ステップ 12 では、各隣接辺を根で最大 1 回スキャンします。根に隣接するスキャンされていない辺がない場合、頂点の割り当てが行われます。 
したがって、ステップ 12 の計算量は $O(max \{p, q\})$ になります。 
結論として、アルゴリズム 3.2 の計算量は $O(max \{p, q\})$ になります。

3.5 幅優先探索

次のセクションと第 4 章で説明するように、幅優先探索 (通常は bfs と略されます) は、
多くのグラフアルゴリズムで役立つもう1つのツールです。幅優先探索は、
ある頂点から開始して、グラフまたは有向グラフの頂点を体系的に訪問します。
$G$ の $r$ (根とも呼ばれます)。 根は最初のアクティブな頂点です。
検索中のどの段階でも、現在アクティブな頂点に隣接するすべての頂点がまだ訪問されていない頂点を探すためにスキャンされます。
つまり、未訪問の頂点に対して「広範な」検索が実行されます。 
頂点に初めてアクセスするたびに、(達成する目標に応じた何らかのルールに従って) ラベルが付けられ、
キューの最後尾に追加されます。 この検索では、スタックではなくキューが使用されることに注意してください。
現在アクティブな頂点は、キューの先頭にある頂点です。 

近隣が訪問されるとすぐに削除され、キューから削除されます。 
キューが空で、グラフまたは有向グラフのいくつかの頂点がまだ訪問されていない場合は、未訪問の頂点を選択し、
ラベルを割り当ててキューに追加します。 グラフのすべての頂点を訪問すると、検索は完了します。

ここで、BFS 中のグラフの頂点のラベル付けの 1 種類について説明します。
$G$ がその隣接リストによって表されるグラフであると仮定します。
最初は、$G$ のすべての頂点は、その隣接リストによって表されるグラフです。 最初に、$G$ のすべての頂点には 0 のラベルが付けられます。
まず、$r$ にラベル 1 を割り当て、$r$ をキュー $Q$ に配置します。