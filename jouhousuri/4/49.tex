\mysection{49ページ目翻訳}

ステップ4が発生するたびに、最大で$p-1$の減算が行われ、
つまり、ステップ4では$O(p)$の演算が行われる。
ステップ2 ~ 4では、バブルソートアルゴリズムを使用する場合、
$O(p^2)$個の比較が必要である。ステップ2-4は$p$回繰り返すことができるので、
このアルゴリズムの複雑さは$O(p^3)$となる。
もちろん、複雑度が低次のソートアルゴリズムを使用すれば、
アルゴリズム2.6の複雑度は改善されるだろう。

2.4 NP-completenessの導入

2.1節で指摘したように、すべての問題は扱いやすいか扱いにくいかのどちらかである。
しかし、多くの問題では、その問題がこの2つのうちどちらに属するかはわからない。
このような問題の中に、「NP完全」問題のクラスがある。
これらの問題は、これらの問題の1つを解く効率的なアルゴリズムが、
これらの問題のすべてに対する効率的なアルゴリズムを保証するという意味で、
すべての問題が本質的に同じである。つまり、これらの問題のうち1つが難解であることが知られている場合、
他の問題も同様である。なお、これはNP完全問題の性質に過ぎず、定義ではない。