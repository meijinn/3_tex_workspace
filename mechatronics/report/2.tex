\mysection{演習3}
\subsection{実行プログラム}
実行プログラムをソースコード\ref{s4}に示す。
\begin{lstlisting}[caption=演習3のプログラム,label=s4]
#include <stdio.h>
#include <dos.h>
#include "v25.h"
#include "ms.h"
 
int main(){
  pokeb( _V25BASE, _PMC2, 0x00 ); /*P20-P27 ポートモード*/
  pokeb( _V25BASE, _PM2, 0x00 );  /*P20-P27 出力ポート  */

  while(1){
      ms_led2( 0xff ); /* 全LEDの点灯 */
      ms_wait( 1000 ); /* 待ち時間 1s */
      ms_led2( 0x00 ); /* 全LEDの消灯 */
      ms_wait( 1000 ); /* 待ち時間 1s */
    }
  return 0;
}
\end{lstlisting}

\mysubsection{実行結果}
ポート2のLEDが点滅する。

\mysection{演習4}
\subsection{実行プログラム}
実行プログラムをソースコード\ref{s5}に示す。
\begin{lstlisting}[caption=演習4のプログラム,label=s5]
int main(){
  pokeb( _V25BASE, _PMC2, 0x00 ); /*P20-P27 ポートモード*/
  pokeb( _V25BASE, _PM2, 0x00 );  /*P20-P27 出力モード*/
 
  while(1){
    int i;
    int a = 0x01;
    for(i=0;i<8;i++)
    {
      ms_led2(a);
      ms_wait(1000);
      a = a << 1;
    }
  }
  return 0;
}
\end{lstlisting}

\subsection{実行結果}
LED0からLED7の順に点灯した後、この一連の動作を繰り返す。

\mysection{演習5}
\subsection{実行プログラム}
実行プログラムをソースコード\ref{s6}に示す。
\begin{lstlisting}[caption=演習5のプログラム,label=s6]
#include <stdio.h>
#include <dos.h>
#include "v25.h"
#include "ms.h"
  
int main(){
    int i;

    pokeb( _V25BASE, _PMC1, 0x20 ); /* P15:TOUT 出力*/
    for( i=0; i<3; i++){
        ms_beep( 440, 500 ); /* 440Hz, 0.5s */
        ms_wait( 500 ); /* 待ち時間 0.5s*/
    }
    ms_beep( 880, 1000 );
    return 0;
}
\end{lstlisting}

\mysubsection{実行結果}
440Hzの音が0.5秒毎に3回鳴った後、880Hzの音が1秒間鳴る。

\mysection{演習6}
\subsection{実行プログラム}
実行プログラムをソースコード\ref{s7}に示す。
\begin{lstlisting}[caption=演習6のプログラム,label=s7]
#include <stdio.h>
#include <dos.h>
#include "v25.h"
#include "ms.h"
  
int main(){
  int i;
  int f;
  int ms;

  pokeb( _V25BASE, _PMC1, 0x20 );
  while(1){
    scanf("%d %d", &f, &ms);
    ms_beep( f, ms ); /* 440Hz, 0.5s */
  }
  return 0;
}
\end{lstlisting}

\mysubsection{実行結果}
JTWのプロンプト上でキーボードから周波数\si{[Hz]}(スペース)時間\si{[s]}を入力して、
エンターキーを入力すると、指定した数値に応じた音がブザーから鳴る。