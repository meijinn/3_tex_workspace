\mysection{12ページ目翻訳}

1.4 サブグラフ\\
グラフ$H$は、$V(H) \subseteq  V(G)$と$E(H) \subseteq E(G)$のとき、グラフ$G$の補グラフとなる 図1-11のグラフ$H$は$G$のサブグラフとなる。
$wy \in E(F)$であるが、$wy \in E(G)$であるので、グラフ$F$は$G$の補グラフではない。

ある種の部分グラフは、あまりに頻繁に出現するため、特別な名前をつけている。
グラフ$G$の頂点集合を$S$とすると、$S$によって誘導される部分グラフは、頂点集合$s$を持つ$G$の最大部分グラフであり、
$S$で示される、すなわち。$S$は$S$内の2つの頂点に結合する$G$の辺を正確に含む。

グラフ$G$の部分グラフ$H$は、頂点誘導部分グラフ、または単に、誘導部分グラフ
図1-11のグラフ$H$が$G$の部分グラフであることは既に述べたが、$x, w \in V(H)$と$xw \in E(G)$であるが、$xw$は$E(H)$であるので、$H$は誘導部分グラフではない。
一方、図1-11のグラフJは、$G$の誘導部分グラフであり、実際には、$$J= \langle {\{v, w, x, y\}} \rangle $$とする。


$G$をグラフとする。$G$の頂点の適切な部分集合$S$の削除は、$S$にないGの頂点と、$S$にある頂点に入射しない$G$の辺を含む部分グラフである。
この部分グラフを$G - S$と表記する。$S$が1つの頂点$v$からなる場合、$G - {v}$の代わりに$G - v$と書く。
図1-12は、グラフ$G$と、$G - v$と$G - \{u, v\}$のグラフを示したものです。$H$がグラフ$G$の誘導部分グラフである場合、
$G$から頂点の部分集合(おそらく空)を削除することによって得られる。