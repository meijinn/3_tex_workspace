\mysection{13ページ目翻訳}

次に、誘導部分グラフの辺の類推を考えてみよう。
グラフ$G$の辺の集合(空でない)を$X$とすると、
$X$によって誘導される部分グラフは、辺集合$X$を持つ$G$の最小グラフであり、$\langle X\rangle $で示される、
すなわち、$\langle X\rangle $は、$X$の1辺に接続する$G$の頂点からなる。

グラフ$G$の部分グラフ$H$は、$G$の辺のある空でない集合$X$に対して$H=\langle X\rangle $であるとき、辺誘導部分グラフである。
図1-13では、$u$は$H_1$の辺に入射しないので、グラフ$H_1$は$G_1$の辺誘導部分グラフではない。しかし、$F_1$、$J_1$はいずれも$G_1$の辺誘導部分グラフであり、
実際には、
$$F_1 = \langle \{uv, uw, wy, yx, xv\}\rangle $$
と
$$J_1 = \langle \{vx, xy, yw\} \rangle$$
とする。


グラフ$G$の部分グラフ$H$は、$V(H)=V(G)$であれば、$G$の全域部分グラフとなる。
図1-13のグラフ$F_1$と$H_1$は$G_1$の全域部分グラフであるが、
$u \in V(G_1) - V(J_1)$ のため、$J_1$は$G_1$の全域部分グラフではない。 

$X$をグラフ$G$の辺の集合とすると、$G - X$は$E(G)$から$X$の辺を削除して得られる$G$の全域部分グラフである。
実際、$H$がグラフ$G$の全域部分グラフであるのは、
$H=G- X$のときだけであり、$X=E(G)-E(H)$である。
$e$がグラフ$G$の辺である場合、$G-\{e\}$ではなく、$G-e$と表記する。
図1-13のグラフ$G_1, F_1, H_1$について、
$F_1=G_1 - vw$, $H_1=G_1 - \{uv, uw\}$とする。

$G$を、$u_i, v_i (i = 1,2,\ldots, n)$を$G$の非隣接頂点の組とするグラフとすると、
$G + {u_1v_1, u2v2, \ldots, u_n, v_n}$は$G$は、集合$\{u_1v_1, u_2v_2, \ldots, u_nv_n\}$に辺を追加することで得られるグラフである。

$u$と$v$が非接続である場合
グラフ$G$の頂点は、$G + \{uv\}$ではなく、$G + uv$と書きます。
例えば、図1-13のような感じです。$G=F+yw, G_1=H+\{uv, uw\}$となる。


対象となるグラフが頂点集合と辺集合で記述される場合は、必ず「グラフの部分グラフ」という概念がうまく定義されます。
しかし、グラフはダイアグラムでも記述できるため、そのような場合は別の定義が必要である。
例えば、図1-14のグラフ$H$は$G$の補グラフであろうか。

これらの図で定義されるグラフ$H$と$G$に対して、$H$と$G$の頂点をラベル付けして、
$H$が先に説明した意味で$G$のサブグラフになることが可能であれば、$H$を$G$のサブグラフと呼ぶことにする。
誘導部分グラフについても同様である。

図1-14では、$H$と$G$にラベルを付けてそれぞれ$H_1$と$G_1$を生成することができるので、
$H$は$G$のサブグラフであることがわかりました。

実際、$H$は$G$の誘導部分グラフである。$H_1$のラベルはこの事実を示していないが、$H_2$にはこの事実がある。
グラフ$F$も$G$の部分グラフであるが、$G$の誘導部分グラフではない。