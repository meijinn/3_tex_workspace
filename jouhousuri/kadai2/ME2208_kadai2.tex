\documentclass[10pt,a4j,dvipdfmx]{jarticle}
%---------------------------------------------------
\usepackage{hyperref}
\usepackage{pxjahyper}
\usepackage{bm}
\usepackage{graphicx}
\usepackage{amssymb,amsmath}
\usepackage{ascmac}
\usepackage{float}
\usepackage{setspace}
\usepackage[dvips,usenames]{color}
\usepackage{colortbl}
\usepackage{algorithm}
\usepackage{algorithmic}
\usepackage{setspace}
\usepackage{subfigure}
\usepackage{here}
\usepackage[deluxe,bold]{otf}
\usepackage[haranoaji]{pxchfon}
\usepackage{redeffont}
\usepackage{listings,jvlisting} %日本語のコメントアウトをする場合jvlisting(もしくはjlisting)が必要
\usepackage{booktabs}
\usepackage{siunitx}
\usepackage{fancybox}
\usepackage{tikz}
%---------------------------------------------------
% \definecolor{bl}{rgb}{0.94,0.97,1}
% \definecolor{gr}{rgb}{0.5,0.5,0.5}
% \makeatletter
% \def\section{\newpage\@startsection {section}{1}{\z@}{2.3ex plus -1ex minus -.2ex}{2.3 ex plus .2ex}{\Large\bf}}
% \makeatother
%---------------------------------------------------
\setlength{\textwidth}{160truemm}
\setlength{\textheight}{240truemm}
\setlength{\topmargin}{-14.5truemm}
\setlength{\oddsidemargin}{-0.5truemm}
\setlength{\headheight}{0truemm}
\setlength{\parindent}{1zw}
\setlength{\abovedisplayskip}{-2pt} % 数式上部のマージン
\setlength{\belowdisplayskip}{-2pt} % 数式下部のマージン
%---------------------------------------------------
\setstretch{1.2}
%---------------------------------------------------
\renewcommand{\subfigtopskip}{5pt}	% 図の上の隙間。上図の副題と下図の間。
\renewcommand{\subfigbottomskip}{0pt} % 図の下の隙間。副題と本題の間。
\renewcommand{\subfigcapskip}{-6pt}	% 図と副題の間
\renewcommand{\subcapsize}{\scriptsize} % 副題の文字の大きさ
\newcommand{\mysection}[1]{\vspace{-20pt}\section{#1}}
\newcommand{\mysubsection}[1]{\vspace{-20pt}\subsection{#1}}
\newcommand{\mysubsubsection}[1]{\vspace{-10pt}\subsubsection{#1}}
\renewcommand{\lstlistingname}{ソースコード}
\newcommand{\tsuyo}[1]{\textbf{\textgt{#1}}}
%---------------------------------------------------
% ヘッダーとフッターの設定
\usepackage{fancyhdr}
\rhead{ME2208\CID{8705}橋尚太郎}
\chead{}
\lhead{情報数理工学 課題2 2章と3章の問題}
\cfoot{\thepage}

\rfoot{}
\begin{document}
%---------------------------------------------------
\setlength{\abovedisplayskip}{1.5pt} 
\setlength{\belowdisplayskip}{0pt}
%---------------------------------------------------
%ここからソースコードの表示に関する設定
\lstset{
  basicstyle={\ttfamily},
  identifierstyle={\small},
  commentstyle={\smallitshape},
  keywordstyle={\small\bfseries},
  ndkeywordstyle={\small},
  stringstyle={\small\ttfamily},
  frame={tb},
  breaklines=true,
  columns=[l]{fullflexible},
  numbers=left,
  xrightmargin=0zw,
  xleftmargin=3zw,
  numberstyle={\scriptsize},
  stepnumber=1,
  numbersep=1zw,
  lineskip=-0.5ex
}
%ここまでソースコードの表示に関する設定
%---------------------------------------------------

\thispagestyle{empty}
\begin{spacing}{1}

\begin{center}
{\Large 明石工業高等専門学校専攻科 \\[1truecm]
情報数理工学 課題2} \\[3.5truecm]
\huge \tsuyo{2章と3章の問題} \\
% \LARGE Digital Ociloscope and Waveform Processing\\
[4truecm]
\Large ME2208 \CID{8705}橋 尚太郎 \\
(機械・電子システム工学専攻2年) \\[1truecm]
提出年月日:\today
\end{center}

% \begin{center}
% \vspace*{3.5cm}
% {\Huge トランジスタ回路の設計}\\
% \vspace*{12.5cm}
% {\Large 電気情報工学科4年後期}\\
% {\Large 電気電子工学実験II}\\
% \vspace*{1cm}
% \end{center}

\end{spacing}
\clearpage
\pagenumbering{arabic}
\pagestyle{fancy}
\setlength{\headheight}{5truemm}

\mysection{Prblem Set 1.2}

\begin{description}
  \setlength{\parskip}{0cm} % 段落間
  \setlength{\itemsep}{0cm} % 項目間
  \item[8 (a) 回答] Tomは、3 回握手をした。
  \item[8 (b) 回答] Tomの妻は、3 回握手をした。
\end{description}

問題の解を与えるグラフを下図に示す。
Coupleは、夫婦の内、夫、妻のどちらかを表す。
\\


\begin{tikzpicture}[every node/.style={circle,draw}]
  \node (A) at (12.5, 2) {Couple 1};
  \node (B) at (7, 0) {Couple 2};
  \node (C) at (1.5, 2) {Couple 2};
  \node (D) at (0, 7) {Couple 3};
  \node (E) at (2.5, 11) {Couple 3};
  \node (F) at (7, 14) {Tom's Couple};
  \node (G) at (10.5, 11) {Tom's Couple};
  \node (H) at (14, 7) {Couple 1};
  \foreach \u \v in {A/B, A/C, A/D, A/E, A/F, A/G, B/D, B/E, B/F, B/G, D/F, D/G}
      \draw (\u) -- (\v);
\end{tikzpicture}

\mysection{Problem Set 1.8}

\begin{description}
  \setlength{\parskip}{0cm} % 段落間
  \setlength{\itemsep}{0cm} % 項目間
  \item[8 回答] $p=p_1\cdot p_2$, $q=p_1\cdot q_2+p_2 \cdot q_1$
\end{description}

\end{document}