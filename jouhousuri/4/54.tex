\mysection{54ページ目翻訳}

SAT のインスタンス $I$ から 3SAT のインスタンス $I'$ への変換が多項式変換であることを示すことはまだ残っています。
$I$ に $m$ 個の節と $n$ 個の変数があるとします。 この場合、$I'$ には最大 $(2n-2)m$ 個の節が含まれます。 
これは、$I$ の各節には最大 $2n$ 個のリテラルがあり、したがって最大 $2n-2$ の新しい節で置き換えられるという事実からわかります。
さらに、$I'$ には最大でも $n+(2n-3)m$ 個の変数があります。

P 問題のクラスは、すべての扱いやすい決定問題で構成される集合であり、
この集合を $P$ で表します。$P$ 内のすべての問題も NP に属します。 推測の段階は実際には冗長です。 
単に問題を解決して証明書を無視するだけでチェック段階を効率的に実行できるため、推測として空の証明書で十分です。
NP $\subseteq$ P かどうか、つまり NP = P かどうかは、おそらく今日の主題における最も重要な未解決 (決定) 問題です。 
NP に属する決定問題に限定していることに言及しておく必要があります。 $\pi_1 \propto \pi$ のような NP 完全問題
$\pi_1$ が存在する決定問題 $\pi$ は、NP 困難と呼ばれます。 
したがって、NP 困難問題は、少なくとも NP 完全問題と同じくらい解決が困難です。

2.5 貪欲なアルゴリズム

このセクションでは、共通の特性を持つアルゴリズムのクラスについて説明します。 
貪欲なアルゴリズムとは、その選択のその後の影響に関係なく、各ステップで可能な限り最善の選択を行うアルゴリズムです。 
後で、グラフ理論における貪欲アルゴリズムの非常に強力な例に遭遇します。 
ここでは、貪欲アルゴリズムの簡単な例について説明します。

旅行者が海外旅行のためにスーツケースに荷物を詰めています。
彼女はいくつかの品物を持ち歩きたいと考えていますが、スーツケースごとに $r$ ポンドの重量制限があります。
彼女が梱包したい $n$ 個のアイテム $1、2、\dots、n$ があります。 
各項目 $i$ は重み w(i) と値または重要度係数 $v(i)$ を持ちます。 
彼女の問題は、最大でも $r$ ポンドまで梱包し、梱包された品物の価値を最大化することです。
これはナップザック問題とも呼ばれます。

この問題を解決するために、貪欲なアルゴリズムを使用します。
項目の値が昇順でないように項目がリストされているとします ($v(1) \leq v(2) \leq \cdots \leq v(n)$)。 
旅行者はまず、重量が最大でも $r$ ポンドである最も価値のある品物を選択します。
さらにアイテムを追加すると重量制限を超えるまで、この方法を続けます。 
このアルゴリズムを正式に紹介します。