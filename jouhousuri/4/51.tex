\mysection{51ページ目翻訳}

(i) と (ii) の両方が成立する場合、非決定的アルゴリズムが $\pi$ を解決すると言います。 
決定問題 $\pi$ を解く非決定的アルゴリズムのチェック段階を (効率的に) 実行できるコンピューターまたはマシンがあると仮定します。
また、これらのコンピュータが無制限に存在すると仮定します。 $\pi$ のインスタンス $I$ に対して考えられる各証明書 $C(I)$ が少なくとも
1 回推測され、これらのコンピューターのいずれかへの入力として $I$ とともに提供された場合、$\pi$ は 効率的に解決できるようになります。
グラフ同型問題と上記の証明書では、$V(G_1)$ から $V(G_2)$ までの $p(G_1)!$ 全単射関数があるため、$p(G_1)!$ の可能な証明書が存在します。

NP 問題のクラスは、合理的な符号化スキームの下で非決定的アルゴリズムによって解決できるすべての決定問題のクラスとして定義されます。 
NP という用語は、非決定的多項式を表します。 NP 問題のセットは NP で示されます。

$\pi_1$ のすべてのインスタンス $I$ をインスタンスに変換する、多項式変換と呼ばれる効率的なアルゴリズム $A$ が存在する場合、
問題 $\pi_1$ は問題 $\pi_2$ に多項式変換できると言います。 $\pi_2$ の $A(I)$ なので、$I$ IN $Y_{\pi_1}$ となり、$pi_1$ のすべてのインスタンス $I$ が
$\pi_2$ のインスタンス $A(I)$ に変換されます。 $A(I) \in Y_{\pi_2}$ がある場合に限り、$I \in Y_{\pi_1}$ になります。 $\pi_1$ を $\pi_2$ に多項式変換できる場合は、
$\pi_1$ inif $\pi_2$ と書き、$\pi_1$ を $\pi_2$ に多項式に変換できるとします。

$\pi_1$ が問題であるとします。「与えられたグラフ $G$ に対して、$V(G)$ には、$S$ の 2 つの頂点が隣接しないように、
$k$ 個の頂点のサブセット $S$ が含まれていますか?」 $\pi_2$ を問題とします: 「与えられたグラフ $G$ に対して、$V(G)$ には $S$ の 
2 つの頂点がすべて隣接するように $k$ 個の頂点のセット $S$ が含まれていますか?」 $\pi_1$ inif $\pi_2$ は、$\pi_1$ のすべてのインスタンス $G$ を $\pi_2$ のインスタンス、
つまり $Gb$ に多項式変換できるためです。
$G$ の次数が $p$ の場合、この変換には $O(p^2)$ ステップが必要です。

NP 完全問題のクラスは、NP 内の他のすべての問題が $\pi$ に多項式変換できるという特性を持つすべての 
NP 問題 $\pi$ で構成されます。 NP 完全問題を定義したので、そのような問題が存在するかどうかを尋ねることもできます。 
結局のところ、NP のすべての問題が多項式で NP に還元できるという性質に問題が生じるのはなぜでしょうか? 
しかし、1971 年に Cook[3] は最初の NP 完全問題の存在を確立しました。 彼は、次に説明する「充足可能性」問題が NP 完全であることを示しました。

ブール変数は、値 true (T) または false (F) を取ることができる変数です。 $x_1、x_2、\dots、x_n$ をブール変数とします。 
ブール変数 $x_i$ の否定 $x_i b$ もブール変数で、$x_i$ が $F$ の場合に限り $T$ になります。リテラルは、変数 $x_i$ またはその否定 $x_i b$ のいずれかです。 
句はリテラルのセットです。 少なくとも 1 つのリテラルが変数 $x_1, x_2, \dots, x_n$ に対する真理値を持っている場合、
節は変数 $x_1, x_2, \dots, x_n$ への真理値の特定の代入に対して満たされていると言います。 少なくとも 1 つのリテラルが真理値 $T$ を持つ場合。
節のセットは、各節が満たされるように変数 $x_1、x_2、\cdots、x_n$ に真理値が割り当てられている場合に満たされます。 
直感的には、各節のリテラルの間に or を配置し、節の間に and を配置することを考えてください。 次に、この複合ステートメントを true にする、
変数 $x_1、x_2、\cdots、x_n$ への真理値の割り当てがあるかどうかを判断します。