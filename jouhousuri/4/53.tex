\mysection{53ページ目翻訳}

これらの節が満足できる場合、節 A(i) は頂点 $i$ が少なくとも 1 つの $V_j$ に属することを保証し、
節 $B(i)、C(i)$、および $D(i)$ は次のことを保証します。 各頂点 $i$ は最大 1 つの $V_j$ に属します。 
したがって、これらの条項は合わせて、それぞれの

各辺 $e = uv$ および各 $j = 1、2、3$ について、次のようにします。
$$F(e,j) = \{\overline{x}_{u,j}、\overline{x}_{v,j}\}$$。

これらの節が満足できる場合、$G$ の隣接する 2 つの頂点が同じ $V_j$ に属さないことが保証されます。
したがって、上記の 54 の節が満たされる場合に限り、同じ $V_j$ に属する 2 つの頂点が隣接しないように、
$V(G)$ を 3 つのセット $V_1、V_2、V_3$ に分割できます。

1971 年に最初の NP 完全問題の存在が確立されて以来、豊富な NP 完全問題が呼び出されてきました。
これらの問題の多くについての良い情報源は、Garey と Johnson の本です [7]。 
NP 内の $\pi$ 問題が、NP 内のすべての $\pi''$ に対して $\pi' \propto  \pi$ となるような NP 完全問題であることを示します。 したがって、$\pi$ は NP 完全です。

2 番目に特定された NP 完全問題という栄誉は、3 つの充足可能性問題に与えられました。 
または 3SAT: それぞれ最大 3 つのリテラルを含む特定の文節のセットは満たされますか? 
前述のアプローチを適用して、この問題が NP 完全であることを示します。

定理2.2。 3SAT は NP 完全です。

証拠。
SAT は NP の問題であるため、3SAT も同様です。
ここで、SAT inf 3SAT を示します。 $I$ を SAT のインスタンスとします。
少なくとも 4 つのリテラルを持つ各句を、それぞれに 3 つのリテラルが正確に含まれる
句のコレクションに置き換えます。

$x_1^*, x_2^*, \dots, x_k^*$ が $C$ を満たすように $x_1, x_2, \dots, x_k$ に真理値を
代入するとします。 次に、少なくとも 1 つの $r (1 \leq r \leq k) について、x_r^* = T$ となります。 
$z_1^*, z_2^*, \dots, z_k^*$ を $z_1, z_2, \dots, z_k$ への真理値の代入とし、
$z_i^* = T for 1 \leq i \leq r とします。 -2$ および $z_i^* = F for r-2 < i \leq k-3$。
このようにリテラル $x_i (1 \leq i \leq k)$ および $z_j (1 \leq j \leq k-3)$ に真理値を代入すると、
新しい各節の少なくとも 1 つのリテラルが真理値を持ちます。 $T$.

逆に、(2.3) の条件が満たされるとします。 
次に、少なくとも 1 つの $x_i$ に真理値 $T$ が割り当てられます。
そうでない場合は、$\{x_1, x_2, z_1\}$ が満たされるため、
$z_1$ には真理値 $T$ を割り当てる必要があります。
このように続けると、$z_i$ はすべての $i (1 \leq i \leq k-3)$ に対して真理値 $T$ を持たなければならないことがわかります。
ただし、$\{x_{k-1}, x_k, \overline{z}_{k-3}\}$ 節は満たされません。

