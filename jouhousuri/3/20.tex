\mysection{20ページ目翻訳}

1.6 連結グラフ

本節では、グラフの構造についての学習を始めます。
グラフ$G$における歩道とは、交互に並ぶ

$$W: v_0, e_1, v_1, e_2, v_2,\dots, v_{n-1},e_n, v_n (n\geq 0)$$

頂点と辺のうち、頂点で始まり頂点で終わるもので、
$i = 1, 2,\cdots, n$ に対して $e_i = v_{i-1}v_i$ となるものです。

$W$は$v_0$から始まり$v_n$で終わるので、
$W$を$v_o-v_n$道と呼ぶ。また、$W$は$n$個の(必ずしも異なる)辺に
出会うので、$W$は長さ$n$を持つとも言われます。
長さ0の歩道を自明な歩道と呼ばれます。
図1-17のグラフ$G$において

$$u, e_1, v, e_3, w, e_6, x, e_4, u, e_1, v, e5, y$$

は長さ6の歩道であり、歩道に現れる頂点が歩道の辺を決めるので、
辺は省略することができます。したがって、辺(1.2)はより簡単に
$u, v, w, x, u, v, y$と表現することができます。

グラフ理論でよく登場する歩道には、いくつかの特殊な種類があります。
小道は辺が繰り返されない歩道であり、
道は頂点が繰り返されない歩道です。したがって、
道は小道ですが、すべての小道が道であるわけではありません。
図1-17のグラフ$G$において、$x, w, v, u, w, y$という歩道は
道ではない小道であり、$u, x, w, v, y$は道です。

\begin{description}
  \item[定理1.3] グラフ内のすべての$u-v$歩道は、$u-v$道を含む。\\
  証明: グラフ$G$の$u-v$歩道を$W$とする。$u=v$とすると、
  自明な$u-u$道が結果を与えるので、$u \neq v$と仮定すればよい。
  $Wを : u = uo, u1, ... u_n=v$(一つの頂点が複数のラベルを受け取ることもありうる)と仮定する。
  $W$に$G$の頂点が2回以上登場しない場合、$W$自体が$u-v$道となる。
  そうでない場合は、$W$に2回以上現れる$G$の頂点が存在する。
  $u_j=u_j$となるような$i$と$j$を異なる正整数、$i<j$とする。
  $W$から、$u_i, u_{i+j}\cdots u_{j-1}$の項を削除すると、
  長さが$W$の長さより小さい$u-v$歩道$W_1$が得られる。
  $W_1$に$G$の頂点が2回以上出現しない場合、$W_1$は$u-v$の道である。
  それ以外の場合は、最終的に$u-v$経路に到達するまで、上記の手順を適用する。
\end{description}