\mysection{6ページ目翻訳}

グラフ$G$の頂点$v$の次数は、$v$に隣接する辺の数であり、$G$の次数が$p$で$v$が$G$の頂点である場合、
$0 \leq \deg v \leq p-1$となる。次数0の頂点は\textbf{\textgt{孤立点}}と呼ばれ、次数1の頂点は\textbf{\textgt{端点}}と呼ばれる。
頂点は、その次数が偶数か奇数かによって\textbf{\textgt{偶数}}か\textbf{\textgt{奇数}}かが決まる。
例えば、図1-6のグラフ$G$は、$v_1, v_2$ と $v_5$の3つの偶数頂点と、$v_3、v_4$の2つの奇数頂点を持ち、$v_5$は孤立し、$v_3$は端点である。

図1-6のグラフ$G$は、次数5、サイズ4、そして
$$\sum_{i = 1}^{5}  \deg v_i = 8$$

つまり、すべての頂点の度数の和は8であり、このグラフでは、この和はグラフの大きさの2倍である。
この観察は常に真であり、この結果は一般に "グラフ理論の第一定理" と呼ばれている。

\begin{description}
  \item[定理1.1] $G$を次数$p$、サイズ$q$のグラフとし、$V(G) = \{ v_1, v_2, \ldots, v_p \}$とすると、
  $$\sum_{i= 1}^{p}  \deg v_i = 2q $$である。
\end{description}